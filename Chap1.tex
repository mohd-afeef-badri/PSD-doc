\chapter{Introduction} 

\section{Introduction} 
PSD acronym for Parallel Solid/Structural/Seismic Dynamics, is a finite elements-based solid mechanics solver with capabilities of performing High Performance Computing (HPC) simulations with billions of unknowns. The kernel of PSD is wrapped around FreeFEM for finite element discretization, and PETSc for linear algebra/Preconditioning. PSD solver contains straightforward supports for static or dynamic simulations with linear  and nonlinear solid mechanics problems. Besides these hybrid-phase field fracture mechanics models have also been incorporated within PSD. For dynamics the generalized-$\alpha$ model  for time discretization is used, this models enable straightforward use of Newmark-$\beta$, central difference, or HHT as time discretization. PSD uses sate-of-the art domain-decomposition paradigm via vectorial finite elements for parallel computing and all solvers are  proven to scale quasi-optimally. PSD has proven scalability uptill 13,000 cores with largest problem solved containing over 5 Billion unknowns. 

Besides the parallel suite, PSD also includes a sequential solver which does not require \href{https://www.mcs.anl.gov/petsc/}{PETSc}.
PSD works for two and three dimensional problems only. Unstructured
meshes (triangular for 2D and tetrahedral for 3D) are supported in
\href{https://www.ljll.math.upmc.fr/frey/software.html}{MEDIT's}
\lstinline!.mesh! format or \href{http://gmsh.info/}{Gmsh's}
\lstinline!.msh! format. PSD post processing is done via
\lstinline!.pvd! ,\lstinline!.vtk! and \lstinline!.vtu! files of the
\href{https://www.paraview.org/}{ParaView} platform.

{\lstset{
		backgroundcolor=\color{white},
		basicstyle=\small\color{black}\ttfamily,
		morekeywords={git, autoreconf, configure, clone, make, sudo, pull},
		frame=trbl, % draw a frame at the top, right, left and bottom of the listing
		frameround=tttt, % make the frame round at all four corners
		framesep=4pt, % quarter circle size of the round corners
	}
\section{Changelog}

PSD has been maturing and evolving with time, following subsections
present the highlights of some key changes made to each PSD version.

\subsection{Rolling release (2.4)}

\subsubsection{Added}

\begin{itemize}
\tightlist
\item
  New \lstinline!PsdMfrontPrintBehaviourStats()! to probe the mfront
  file and know the input/output.
\item
  New variables added for \lstinline!PsdMfrontHandler()! function
  \lstinline!mfrontStrainTensor, mfrontStateVariable,!.
\item
  New variables added for \lstinline!PsdMfrontHandler()! function
  \lstinline!mfrontExternalStateVariableNames, mfrontExternalStateVariableValues,!.
\item
  More unit tests for mfront plugin.
\item
  New elasto-plastic von Mises non-linear model using mfornt. Via
  argument \lstinline!-problem elasto_plastic! and
  \lstinline!-model von_mises!
\item
  New elasto-plastic tutorial.
\item
  New automake variable \lstinline!--with-dependencies=yes! for
  installing all dependencies (FreeFEM, PETSc, Gmsh, Mfront, MGISq)
\item
  Dependencies can now be installed by PSD.
\end{itemize}

\subsubsection{Changed}

\begin{itemize}
\tightlist
\item
  Moved to FreeFEM 4.10
\item
  Moved to PETSc 3.16.1
\item
  Mfront material handler has a new name:
  \lstinline!mfrontElasticityHandler! is now
  \lstinline!PsdMfrontHandler!
\end{itemize}

\subsubsection{Removed}

\subsubsection{Bug}

\subsection{\texorpdfstring{\href{https://gitlab.com/PsdSolver/psd_sources/-/tree/v2.3}{2.3}
- 2021-09-13}{2.3 - 2021-09-13}}

\subsubsection{Added}

\begin{itemize}
\tightlist
\item
  New argument \lstinline!-useMfront! now activates the Mfront--PSD
  coupling.
\item
  More tests for each plugin and physics, now added in their respective
  folders.
\item
  Support for Mfront and Mgis interface for non-linear material laws
\item
  New plugin \lstinline!mfront! can now be loaded in PSD.
\item
  \lstinline!-withmaterialtensor! now uses Quadrature finite elements to
  build the material tensor.
\item
  New variational formulation for handling Linear mechanics problems,
  Quadrature point wise material tensor is built.
\item
  \lstinline!pseudo_nonlinear! implementation of Linear Elasticity now
  works.
\item
  New \lstinline!Notes! section added in the source repo, to help with
  mathematical and algorithmic reasoning.
\end{itemize}

\subsubsection{Bug}

\begin{itemize}
\tightlist
\item
  Bug in top-ii-vol meshing due to MPI communication removed.
\item
  Bug in \lstinline!pseudo_nonlinear! model for
  Elastodynamics/soildynamics fixed.
\end{itemize}

\subsection{\texorpdfstring{\href{https://gitlab.com/PsdSolver/psd_sources/-/tree/v2.2}{2.2}
- 2021-07-28}{2.2 - 2021-07-28}}

\subsubsection{Added}

\begin{itemize}
\tightlist
\item
  New and more verbose tutorials on fracture mechanics, soil-dynamics.
\item
  Fast and parallel post processing is now performed using
  \lstinline!pvtu! files.
\item
  New PETSc interface in plugins that supports pvtu output.
\item
  Error mechanism to signify wrong PSD flags.
\end{itemize}

\subsubsection{Changed}

\begin{itemize}
\tightlist
\item
  Flag values now donot contain hyphens `-' use underscore instead '\_',
  e.g, \lstinline!linear-elasticity! is now
  \lstinline!linear_elasticity!
\item
  4 CPU procs are now used for \lstinline!make check!, user can control
  this by \lstinline!make check NP=USER_PROCS!.
\item
  Moved to FreeFEM 4.9 and PETSc 3.15.0 .
\item
  Moved to GitLab for hosting the repository.
\item
  New checks for wrong flag. Now if wrong flag or values is entered PSD
  will give error.
\item
  Boolean flags now also accept value
  1\textbar{}0\textbar{}yes\textbar{}no\textbar{}on\textbar{}off\textbar{}true\textbar{}false
  for turning on or off.
\end{itemize}

\subsection{\texorpdfstring{\href{https://gitlab.com/PsdSolver/psd_sources/-/tree/v2.1}{2.1}
- 2021-01-27}{2.1 - 2021-01-27}}

\subsubsection{Added}

\begin{itemize}
\tightlist
\item
  New accurate force calculations via matrix-vector product: new flag
  \lstinline!-getreactionforce!.
\item
  New flag \lstinline!-reactionforce! variational-based \textbar{}
  stress-based to get reaction force on a surface.
\item
  New flag \lstinline!-plotreactionforce! to activate real time pipe
  plotting using GnuPlot.
\item
  More verbos info on \lstinline!-help!
\item
  New flag \lstinline!-mesh! to provide the name of mesh to
  \lstinline!PSD_Solve!.
\item
  New flag \lstinline!-probe! to postprocess FE variables at a point.
\item
  New flag \lstinline!-crackdirichletcondition! to include a pre-cracked
  Dirichlet in damage models.
\item
  New tests for more advance top-ii-vol partitioning.
\item
  New flag \lstinline!-constrainHPF! to enable constrain conditions in
  hybrid phase-field (WIP).
\item
  New developments in parallel interpolations.
\item
  Tutorials added, use \lstinline!make tutorials! to install
\end{itemize}

\subsubsection{Changed}

\begin{itemize}
\tightlist
\item
  Moved to FreeFEM 4.7-1.
\item
  Moved to PETSc 3.14.
\item
  New version of top-ii-vol v 1.3 support for exascale computing
  (includes new 2D 3D partitioning)
\item
  \lstinline!-fastmethod! now replaced by
  \lstinline!-withmaterialtensor! (this is now inverse of
  \lstinline!-fastmethod!)
\end{itemize}

\subsubsection{Removed}

\begin{itemize}
\tightlist
\item
  Flag \lstinline!-pipegnu! not supported for damage mechanics (to be
  further deprecated from elsto/soildynamics)
\end{itemize}

\subsubsection{Bugs removed}

\begin{itemize}
\tightlist
\item
  Stain vector incorrect numbering in split function of GFP
\end{itemize}

( see hash
\href{https://codev-tuleap.intra.cea.fr/plugins/git/hpcseism/freefem?a=commit\&h=5ec7b882494f71984d07f468b518ec886e942d32}{5ec7b882494f71984d07f468b518ec886e942d32})

\begin{itemize}
\tightlist
\item
  Hybrid phase-field with constrain with wrong update
\end{itemize}

(see has
\href{https://codev-tuleap.intra.cea.fr/plugins/git/hpcseism/freefem?a=commit\&h=20ebfbc3cc58b9f1407658543bf3b239e74bd089}{20ebfbc3cc58b9f1407658543bf3b239e74bd089})

\subsection{\texorpdfstring{\href{https://gitlab.com/PsdSolver/psd_sources/-/tree/v2.0}{2.0}
- 2020-08-18}{2.0 - 2020-08-18}}

\subsubsection{Added}

\begin{itemize}
\tightlist
\item
  New processing via C++, \lstinline!PSD_PreProcess! binary (MAJOR
  CHANGE).
\item
  New solving via shell wrapper \lstinline!PSD_Solve! instead of
  \lstinline!FreeFem++! or \lstinline!FreeFem++-mpi!.
\item
  New examples of using the solver.
\item
  New Pdf documentation containing tutorials, function definitions,
  verbos on PSD C++ library.
\item
  New option \lstinline!-nonlinearmethod!
  (Picard\textbar{}Newton-Raphsons).
\item
  New time discretization option \lstinline!-timediscretization!.
\item
  New point boundary conditions.
\item
  New dummy city mesh and analysis 2D for soil dynamics.
\item
  Automatic identification of FreeFEM and Gmsh during
  \lstinline!./configure!.
\item
  New flags for \lstinline!--with-FreeFEM=! and \lstinline!--with-Gmsh=!
  during \lstinline!./configure!.
\item
  New flag \lstinline!-bodyforceconditions! {[}int{]} to include body
  force.
\item
  New flag \lstinline!-problem! linear-elasticity \textbar{} damage
  \textbar{} elastodynamics \textbar{} soildynamics to define physics.
\item
  New flag \lstinline!-model! to set approximation for damage mechanics
  hybrid-phase-field \textbar{} Mazar \textbar{} pseudo-nonlinear
  \textbar{} Hujeux.
\item
  Better energy splitting included Hybrid phase-field compressibility vs
  tensile energy condition.
\item
  Introduce boundary conditions via \lstinline!-dirichletconditions!
  {[}int{]} flag.
\item
  Introduce point boundary conditions via
  \lstinline!-dirichletpointcondition! {[}int{]} flag.
\item
  Introduce traction boundary conditions via
  \lstinline!-tractionconditions! {[}int{]} flag.
\item
  New folder \lstinline!tests! containing unit-tests for modules.
\item
  New Hujeux law (nonlinear soil law) interface using C++ class (Thanks
  to Evelyne Foerster).
\item
  New pseudo-nonlinear model for solving elastodynamics and soildynamics
  with nonlinear Newton-Raphsons.
\item
  Introduced double couple earthquake source boundary condition for
  soildynamics.
\item
  New flag \lstinline!-doublecouple! force-based \textbar{}
  displacement-based to use double couple source for soildynamics.
\item
  New top-ii-vol parallel meshing via \lstinline!-top2vol-meshing! flag
  (compatible with soildynamics).
\end{itemize}

\subsubsection{Changed}

\begin{itemize}
\tightlist
\item
  Moved to FreeFEM 4.6.
\item
  Moved to PETSc 13.13.
\item
  Moved to C++ for preprocessing.
\item
  Moved to shell wrapper PSD\_Solve for solving.
\item
  Changes to GFP energydecopostion plugin `DecompEnergy\_Op', now
  includes compressibility history.
\item
  Replaced GFPDecompEnergy2D/GFPDecompEnergy3D by a generic 2D/3D
  function
  GFPSplitEnergy(Eps1{[}{]},PsiPlus{[}{]},PsiMinus{[}{]},HistPlus{[}{]},HistMinus{[}{]},par);.
\item
  Postprocessing flag \lstinline!-postprocess! options now support
  \lstinline!u! , \lstinline!v!, \lstinline!a! , \lstinline!uv!,
  \lstinline!ua!, \lstinline!av! or \lstinline!uav!.
\end{itemize}

\subsubsection{Removed}

\begin{itemize}
\tightlist
\item
  No more \lstinline!FFINSTALLDIR! and \lstinline!GMSH! variables during
  \lstinline!make! and \lstinline!make check!.
\item
  No more \lstinline!-plot! flag now handled by
  \lstinline!-postprocess!.
\item
  No more \lstinline!-nonlinear! flag now handled by
  \lstinline!-problem! and \lstinline!-model!.
\item
  No more \lstinline!-bodyforce! flag now handled by an \lstinline!int!
  valued \lstinline!-bodyforceconditions!.
\item
  No more \lstinline!-dynamic! flag now handled by \lstinline!-problem!
  and \lstinline!-model!.
\item
  No more \lstinline!-soildynamic! flag now handled by
  \lstinline!-problem! and \lstinline!-model!.
\item
  No more \lstinline!-quasistatic! flag now handled by
  \lstinline!-problem! and \lstinline!-model!.
\item
  No more \lstinline!-dirichletbc! flag now handled by
  \lstinline!-dirichletconditions!.
\end{itemize}

\subsubsection{Bugs removed}

\begin{itemize}
\tightlist
\item
  Bug in 3D paraxial loading (see hash
  \href{https://codev-tuleap.intra.cea.fr/plugins/git/hpcseism/freefem?a=blobdiff\&h=f23846d41b3eee27b9f2c3b852b9c5e2f90080c5\&hp=0af4cfcc1355b98cc09a94e42c8ec521c6c0bb00\&hb=8fcbe7390e526423cd24b5f0ab1c06899b36c67f\&f=src\%2Fpsd-preprocessor\%2FMacros.hpp}{8fcbe7390e526423cd24b5f0ab1c06899b36c67f})
\end{itemize}

\subsection{\texorpdfstring{\href{https://gitlab.com/PsdSolver/psd_sources/-/tree/v1.8}{1.8}
- 2020-01-21}{1.8 - 2020-01-21}}

\subsubsection{Added}

\begin{itemize}
\tightlist
\item
  New soil dynamic module \lstinline!-soildynamics!
\item
  New paraxial element support in 2D.
\item
  New timeplotting support \lstinline!-timepvd!
\item
  New \lstinline!-postprocess! option for postprocessing \lstinline!u! ,
  \lstinline!v!, \lstinline!a! , or \lstinline!uav!.
\end{itemize}

\subsubsection{Changed}

\begin{itemize}
\tightlist
\item
  Moved to FreeFEM 4.4.2.
\item
  Moved to PETSc 13.12.
\item
  New simpler way of plotting \lstinline!savevtk! in parallel with
  \lstinline!append! flag for iterative solutions.
\item
  VTU files get stored with a date and time stamp.
\item
  New way of maintaining a logfile for all simulations
  (date,time,case,..) in \lstinline!simulation-log.csv!.
\end{itemize}

\subsection{\texorpdfstring{\href{https://gitlab.com/PsdSolver/psd_sources/-/tree/v1.7}{1.7}
- 2019-11-08}{1.7 - 2019-11-08}}

\subsubsection{Added}

\begin{itemize}
\tightlist
\item
  New mesh reordering via Reverse Cuthill-Mackee via
  \lstinline!-useRCM!.
\item
  New quasi-static parallel solver (Extension of B.Masseron \&
  G.Rastiello sequential version).
\item
  New GFP plugin for Mazar's damage update for 2D/3D problems
  \lstinline!GFPMazarsDamageUpdate(...)!.
\item
  New MPI plotting routine \lstinline!plotJustMeshMPI()!.
\item
  New option \lstinline!-fastmethod! to switch back to default
  variational formulation.
\item
  New make flag for compiling on supercomputer.
\end{itemize}

\subsubsection{Changed}

\begin{itemize}
\tightlist
\item
  Changed variational formulation now using
  \(\epsilon(u):A:\epsilon(v)\).
\item
  Using GFP becomes optional \lstinline!-useGFP!.
\item
  Better documentation via \lstinline!.md! and \lstinline!.html! files.
\item
  Better plotting support for \lstinline!PlotMPI()!.
\item
  Moved to FreeFEM 4.4.
\end{itemize}

\subsection{\texorpdfstring{\href{https://gitlab.com/PsdSolver/psd_sources/-/tree/v1.6}{1.6}
- 2019-06-11}{1.6 - 2019-06-11}}

\subsubsection{Added}

\begin{itemize}
\tightlist
\item
  Dynamic linear solver in 2D and 3D parallel/sequential.
\item
  New finite element variable for partition of unity-- for fixing
  integrals.
\end{itemize}

\subsubsection{Changed}

\begin{itemize}
\tightlist
\item
  Better documentation via \lstinline!.md! and \lstinline!.html! files.
\item
  Correct quadrature order for faster computations.
\item
  Major changes/splitting of \lstinline!.script! files.
\end{itemize}

\subsubsection{Removed}

\begin{itemize}
\tightlist
\item
  Removed the \lstinline!BoundaryAndSourceConditions.script! merged with
  \lstinline!ControlParameters.script!.
\end{itemize}

\subsubsection{Bugs}

\begin{itemize}
\tightlist
\item
  Bug in integrals fixed.
\end{itemize}

\subsection{\texorpdfstring{\href{https://gitlab.com/PsdSolver/psd_sources/-/tree/v1.5}{1.5}
- 2019-05-29}{1.5 - 2019-05-29}}

\subsubsection{Added}

\begin{itemize}
\tightlist
\item
  Dynamic linear solver in 2D and 3D sequential.
\item
  New meshes for dynamics tests \lstinline!bar-dynamic.msh!.
\item
  Checking modules \lstinline!make check!.
\item
  Faster sparsity pattern calculations.
\end{itemize}

\subsubsection{Changed}

\begin{itemize}
\tightlist
\item
  Better documentation via \lstinline!.md! and \lstinline!.html! files.
\item
  Major restructuring of the codes.
\item
  Moved to \lstinline!automake! for solver installation.
\item
  Mesh building via \lstinline!make!.
\end{itemize}

\subsubsection{Removed}

\begin{itemize}
\tightlist
\item
  Removed the manufactured solution codes.
\end{itemize}

\subsection{\texorpdfstring{\href{https://gitlab.com/PsdSolver/psd_sources/-/tree/v1.4}{1.4}
- 2019-05-14}{1.4 - 2019-05-14}}

\subsubsection{Added}

\begin{itemize}
\tightlist
\item
  Fully vectorial finite element solver for phase-filed
  \lstinline!-vectorial!.
\item
  New \lstinline!-supercomp! for avoiding xterm issues on super
  computers.
\item
  New \lstinline!MatViz()! function for matrix sparsity visualization.
\item
  Introduced \lstinline!GFP! plugin support (Go Fast Plugins).
\end{itemize}

\subsubsection{Changed}

\begin{itemize}
\tightlist
\item
  Elastic energy decomposition is now optional
  \lstinline!-energydecomp!.
\item
  Force calculation using integrals (Thanks to G.Rastiello).
\end{itemize}

\subsection{\texorpdfstring{\href{https://gitlab.com/PsdSolver/psd_sources/-/tree/v1.3}{1.3}
- 2019-04-08}{1.3 - 2019-04-08}}

\subsubsection{Added}

\begin{itemize}
\tightlist
\item
  New meshes in 2D/3D\lstinline!Notched-plate! ,
  \lstinline!square-crack!, etc.
\item
  New fracture mechanics module.
\item
  New \lstinline!-nonlinear! flag to activate phase-field model for
  brittle fracture.
\item
  New \lstinline!-timelog! for time logging the solver.
\item
  New \lstinline!-pipegnu! for GNUplot piping.
\end{itemize}

\subsubsection{Changed}

\begin{itemize}
\tightlist
\item
  Scripting now performed using \lstinline!.script! files:

  \begin{itemize}
  \tightlist
  \item
    \lstinline!BoundaryAndSourceConditions.script!
  \item
    \lstinline!LinearFormBuilderAndSolver.script!
  \item
    \lstinline!Macros.script!
  \item
    \lstinline!Main.script!
  \item
    \lstinline!VariationalFormulation.script!
  \item
    \ldots{}.
  \end{itemize}
\item
  Move to FreeFEM version 4.0.
\item
  Move to PETSc version 3.11.
\end{itemize}

\subsection{\texorpdfstring{\href{https://gitlab.com/PsdSolver/psd_sources/-/tree/v1.1}{1.2}
- 2019-03-18}{1.2 - 2019-03-18}}

\subsubsection{Added}

\begin{itemize}
\tightlist
\item
  Support for Gmsh's \lstinline!.msh! or Medit's \lstinline!.mesh!
  meshes in folder \lstinline!Meshes!.
\item
  Advance to 3D physics.
\item
  New MPI based parallel solver linear elasticity.
\item
  New approach for solver generation via scripting (PhD thesis MA Badri)
  with \lstinline!scriptGenerator.edp!.
\item
  Integrated Domain decomposition macro (PhD thesis MA Badri).
\item
  Customized \lstinline!.vtk! support for ParaView post-processing.
\item
  New point boundary condition macro
  \lstinline!pointbc(Real[int], fespace, matrix)!.
\item
  New flags for communicating with the solver: \lstinline!-dimension!,
  \lstinline!-plot!, \lstinline!-bodyforce!, \lstinline!-lagrange!, etc.
\end{itemize}

\subsubsection{Changed}

\begin{itemize}
\tightlist
\item
  More advance README.MD.
\item
  Sequential solver now merged within scripting via flag
  \lstinline!-sequential!.
\item
  Move to FreeFEM version 3.62.
\item
  Moved manufactured solutions to \lstinline!validation-test! folder.
\end{itemize}

\subsection{\texorpdfstring{\href{https://gitlab.com/PsdSolver/psd_sources/-/tree/v1.1}{1.1}
- 2019-03-04}{1.1 - 2019-03-04}}

\subsubsection{Added}

\begin{itemize}
\tightlist
\item
  Initial FreeFEM files for sequential linear elasticity in 2D (case of
  constrained bar).
\item
  More cases of manufactured solution for linear elasticity in 2D.
\item
  Added README.MD for explaining the solver.
\item
  ParaView plotting activated.
\end{itemize}

\subsubsection{Changed}

\begin{itemize}
\tightlist
\item
  Moved to Tuleap git hosting from CEA.
\item
  Separate folder of manufactured solutions and the linear elastic
  solver.
\item
  Move to FreeFEM version 3.61.
\end{itemize}

\subsection{\texorpdfstring{\href{https://gitlab.com/PsdSolver/psd_sources/-/tree/v1.0}{1.0}
- 2019-02-15}{1.0 - 2019-02-15}}

\subsubsection{Added}

\begin{itemize}
\tightlist
\item
  Initial FreeFEM files Method of manufactured solution for linear
  elasticity in 2D.
\end{itemize}

\subsection{Version git tags}

\begin{longtable}[]{@{}ll@{}}
\toprule
Version & Git tag\tabularnewline
\midrule
\endhead
\href{https://gitlab.com/PsdSolver/psd_sources/-/tree/v1.0}{1.0} &
8a8ecb2746b7da792073358c60df33bae647f788\tabularnewline
\href{https://gitlab.com/PsdSolver/psd_sources/-/tree/v1.1}{1.1} &
a667e6085ba1f92f8dd619bd40e18f85c593bc0a\tabularnewline
\href{https://gitlab.com/PsdSolver/psd_sources/-/tree/v1.2}{1.2} &
e48b7b3a30c05ad4c343efa6a17fee386031f437\tabularnewline
\href{https://gitlab.com/PsdSolver/psd_sources/-/tree/v1.3}{1.3} &
39f4324550365849852c5264b8d4535aae05e30d\tabularnewline
\href{https://gitlab.com/PsdSolver/psd_sources/-/tree/v1.4}{1.4} &
f51f678630eb9b2fed355e5cedf976ce8b5fa341\tabularnewline
\href{https://gitlab.com/PsdSolver/psd_sources/-/tree/v1.5}{1.5} &
07293ba09a69d3d6a16278220a0b4a7a9f318f96\tabularnewline
\href{https://gitlab.com/PsdSolver/psd_sources/-/tree/v1.6}{1.6} &
f359dd049fb1ddde376e8ad8e5177c663e430418\tabularnewline
\href{https://gitlab.com/PsdSolver/psd_sources/-/tree/v1.7}{1.7} &
aee9bfec868a70b3d9974d7692bc19f9739ab7dc\tabularnewline
\href{https://gitlab.com/PsdSolver/psd_sources/-/tree/v1.8}{1.8} &
2f26292636c7248133e31ae912ee58113de2ef71\tabularnewline
\href{https://gitlab.com/PsdSolver/psd_sources/-/tree/v2.0}{2.0} &
1e1a4d7f10df30d106b52eba1c5caf69e8bc0f36\tabularnewline
\href{https://gitlab.com/PsdSolver/psd_sources/-/tree/v2.1}{2.1} &
8b9d84f25aedbd684eb0f06cdd4ffbbf9a60a0e2\tabularnewline
\href{https://gitlab.com/PsdSolver/psd_sources/-/tree/v2.2}{2.2} &
5e0368f990d505d3bf1960122cb99a23e08567b0\tabularnewline
\href{https://gitlab.com/PsdSolver/psd_sources/-/tree/v2.3}{2.3} &
0744b19fbe7da6d523754092e92f3882b57f0760\tabularnewline
\bottomrule
\end{longtable}

}
\pagebreak

\begin{figure}
    \centering
		\begin{tikzpicture}
		[
		grow cyclic,
		text=white,
		every node/.style={concept, scale=.95},
		level 1/.style={level distance = .1mm,scale=0.95},   
		level 2/.style={level distance = 5mm}  
		]
		\path[mindmap,concept color=black,text=white, scale=0.95]
		node[concept] {\textbf{PSD}}
		[clockwise from=-8]
		child[concept color=orange,clockwise from=26] {
			node[concept] {\textbf{Meshing}}
			[clockwise from=120]
			child { node[concept color=orange] {\textbf{\underline{\scriptsize Gmsh} \\ {\tiny\color{blue} C++} \\{\tiny del2D/3D}\\{\tiny Netgen}}}}
			child { node[concept color=orange] {\textbf{\underline{{\scriptsize Salome} }  \\ {\tiny\color{blue} Python} \\ {\tiny Netgen} \\ {\tiny MeshGems}  }}}
			child { node[concept] {\textbf{\underline{\scriptsize MeshGem} \\ {\tiny\color{blue} C\# -- MPI} \\{\tiny del2D/3D}  }}}
			child { node[concept] {\textbf{ \underline{\scriptsize top-ii-vol} \\ {\tiny\color{blue} C++ -- MPI-I/O} \\{\tiny Laplace based}   }} }			
		}  
		child[concept color=orange,clockwise from=10] {
			node[concept] {\textbf{Partitioning}}
			[clockwise from=20]
			child { node[concept] {\textbf{\underline{\scriptsize ParMETIS}} \\ {\tiny\color{blue} C\#/C++ -- MPI}    } }
			child { node[concept] {\textbf{\underline{\scriptsize PTSCOTCH} \\ {\tiny\color{blue} C\# -- MPI -- POSIX}    }   } }
			child { node[concept] {\textbf{ \underline{\scriptsize top-ii-vol} \\ {\tiny\color{blue} C++ -- MPI-I/O} \\{\tiny Laplace based}   }} }				
		}
		child[concept color=orange] { 
			node[concept] {\textbf{FEM}} 
			[clockwise from=-160] 
			child { node[concept] { \textbf{ \underline{\scriptsize FreeFEM } \\ {\tiny\color{blue} C/C++ -- MPI}   } }}
			child { node[concept] {\textbf{\underline{\scriptsize GFPlib} \\ {\tiny\color{blue} C/C++ -- MPI} \\ {\tiny vectorial FEM}  \\ {\tiny Mech. Laws}       }  }}
		}
		child[concept color=orange] { 
			node[concept] {\textbf{Solving \& precond.} } 
			[clockwise from=-160] 
			child { node[concept] {\textbf{\underline{\scriptsize PETSc} \\{\color{black}\tiny Krylov CG} \\{\color{black}\tiny Jac, AMG}     }}}
			child { node[concept] {\textbf{\underline{\scriptsize MUMPS}  \\{\tiny\color{black} LU decomp.}       }}}
		}
		child[concept color=orange] { 
			node[concept] {\textbf{Post-processing} } 
			[clockwise from=-180] 
			child { node[concept] {\textbf{\underline{\scriptsize ParaView} \\ {\tiny\color{blue} Python -- MPI} \\ {\tiny Pvpython} }}}
		};
	    		\end{tikzpicture}    \caption{PSD app architecture}
    \label{fig:my_label}
\end{figure}


