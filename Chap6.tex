\chapter{Functions and descriptions}
\section{Flags for PSD\_PreProcess}

The {\ttfamily PSD\_PreProcess}  relies heavily on command line flags for user interaction. These flags become a medium of communication between the user and the PSD solver. Three types of flags can be used i)  {\ttfamily int} type : these are integer type flags which expect an integer argument, ii) {\ttfamily string} type : these flags expect a string argument, and iii) {\ttfamily bool} type : these are boolean type flags with no argument. 

\subsection{Integer type flags used for  PSD\_PreProcess}
\begin{conditions*}
  -dirichletpointconditions &  Number of Dirichlet points.  Default \ttfamily 0.\\

  -dirichletconditions      &   Number of Dirichlet boundaries.  Default \ttfamily 0.\\
  
  -bodyforceconditions      &   Number of regions to which body force is applied.  Default \ttfamily 0.\\  

  -tractionconditions       &   Number of Neumann/traction boundaries.  Default \ttfamily 0.\\

  -parmetis\_worker          &   Active when mesh partitioner is parmetis.\\

  -dimension                &  Dimension of problem. 2 for 2D 3 for 3D. Default 2.\\
  
  -lagrange                 &   Lagrange order used for building FE space. Default \ttfamily 1 for P1. \\
\end{conditions*} 

\subsection{String type flags used for  PSD\_PreProcess}
\begin{conditions*} 
  -timediscretization & Time discretization type. Use \ttfamily generalized-alpha | \ttfamily newmark-beta | \ttfamily hht-alpha | \ttfamily central-difference. \\	
  
  -nonlinearmethod & Nonlinear method type. Use \ttfamily Picard | \ttfamily Newton-Raphsons. \\ 			

  -partitioner & Mesh partitioner. Use \ttfamily metis | scotch | parmetis.\\

  -postprocess & Indicate postprocessing quantity. Use \ttfamily u | v | a | phi | uphi | uva.\\

  -doublecouple  & Soil dynamics double couple boundary condition. Use \ttfamily force-based | displacement-based.\\
  
  -reactionforce  &  Reaction force calculation method. Use \ttfamily stress-based | variational-based.\\
  
  -problem     & Interested problem. Use \ttfamily linear-elasticity | damage | elastodynamics | soildynamics.\\

  -model       & Interested model. Use \ttfamily hybrid-phase-field | Mazar | pseudo-nonlinear.	\\		

\end{conditions*}
\subsection{Bool type flags used for  PSD\_PreProcess}
\begin{conditions*} 
  -help         &      Helping message on the terminal.  \\
  			  
  -debug        &      OpenGL plotting routine for displaying solution. \\
  
  -useGFP       & 	  Activate use of GoFastPlugins. A suite of C++ plugins.\\
  
  -timelog      & 	  To setup time logging for various phases of the solver. \\ 

  -useRCM       & 	  Mesh level renumbering via Reverse Cuthill Mckee.\\   
  
  -vectorial    & 	  Generate vectorial space solver for non-linear.  \\ 
  
  -sequential   & 	  To generate a sequential PSD solver. \\
  
  -fastmethod    & 	  Produce a fast solver (more optimized).   \\ 
  
  -pointprobe    & 	  Setup a point probe to record variables.   \\ 

  -energydecomp &     Hybrid phase-field energy decomposition. \\ 

  -top2vol-meshing &  top-ii-vol point source meshing for soil-dynamics. \\
  
  -getreactionforce &  Extraction reactions at surface. \\ 

  -plotreactionforce &  Live pipe plotting using GnuPlot.
  
    


  
\end{conditions*}
\section{Flags as per physics}
\begin{small}

\fbox{
\begin{minipage}[t][11cm]{0.2\textwidth}
  \centering \textbf{Linear Elasticity}
  \begin{scriptsize}
  \begin{conditions*}

  	INT TYPE &   \\ &   \\
  	-dirichletpointconditions &   \\
    -dirichletconditions      &   \\
    -bodyforceconditions      &   \\  
    -tractionconditions       &   \\
    -parmetis\_worker         &   \\
  	-dimension                &   \\    
    -lagrange                 &   \\
     &   \\
     &   \\

 
   STRING TYPE &   \\ &   \\ 
   -partitioner               &   \\
   -postprocess               &   \\
   -problem                   &   \\   		    
     &   \\
     &   \\
     &   \\
     &   \\
     &   \\    
       

   BOOL TYPE &   \\ &   \\
   -sequential                &   \\
   -fastmethod                &   \\
   -timelog                   &   \\
   -useGFP                    &   \\ 
   -useRCM                    &   \\         
   -debug                     &   \\
   -help                      &   \\
       		
  \end{conditions*}
  \end{scriptsize}
\end{minipage}
}\hfill\fbox{
\begin{minipage}[t][11cm]{0.2\textwidth}
  \centering \textbf{Damage Mechanics}
  \begin{scriptsize}
	\begin{conditions*}
		
		INT TYPE &   \\ &   \\
		-dirichletpointconditions &   \\
		-dirichletconditions      &   \\
		-bodyforceconditions      &   \\  
		-tractionconditions       &   \\
		-parmetis\_worker         &   \\
		-dimension                &   \\    
		-lagrange                 &   \\
		&   \\
		&   \\
		
		
		STRING TYPE &   \\ &   \\ 
		-nonlinearmethod           &   \\
		-reactionforce             &   \\
		-partitioner               &   \\
		-postprocess               &   \\
		-problem                   &   \\   		    
		-model                     &   \\
		&   \\
		&   \\
		
		BOOL TYPE &   \\ &   \\
    -plotreactionforce         &   \\		
    -getreactionforce          &   \\ 
		-energydecomp              &   \\				
		-sequential                &   \\
		-vectorial                 &   \\
		-timelog                   &   \\
		-useGFP                    &   \\
        -useRCM                    &   \\ 				   		
		-debug                     &   \\
		-help                      &   \\		


      		
	\end{conditions*}
\end{scriptsize}
\end{minipage}
}\hfill\fbox{
\begin{minipage}[t][11cm]{0.2\textwidth}
 \centering  \textbf{Elastodynamics}
  \begin{scriptsize}
	\begin{conditions*}
		
		INT TYPE &   \\ &   \\
		-dirichletpointconditions &   \\
		-dirichletconditions      &   \\
		-bodyforceconditions      &   \\  
		-tractionconditions       &   \\
		-parmetis\_worker         &   \\
		-dimension                &   \\    
		-lagrange                 &   \\
		 &   \\
		 &   \\
		
		
		STRING TYPE &   \\ &   \\
		-timediscretization        &   \\		 
		-partitioner               &   \\
		-postprocess               &   \\
		-problem                   &   \\   		    
		-model                     &   \\
		&   \\
		&   \\
		&   \\				
		BOOL TYPE &   \\ &   \\				
		-sequential                &   \\
		-timelog                   &   \\
		-useGFP                    &   \\
		-useRCM                    &   \\ 				   		
		-debug                     &   \\
		-help                      &   \\		

		
		
	\end{conditions*}
\end{scriptsize}
\end{minipage}
}\hfill\fbox{
\begin{minipage}[t][11cm]{0.2\textwidth}
  \centering \textbf{Soildynamics}
  \begin{scriptsize}
	\begin{conditions*}
		
		INT TYPE &   \\ &   \\
		-dirichletpointconditions &   \\
		-dirichletconditions      &   \\
		-bodyforceconditions      &   \\  
		-tractionconditions       &   \\
		-parmetis\_worker         &   \\
		-dimension                &   \\    
		-lagrange                 &   \\
		 &   \\
		 &   \\
		
		
		STRING TYPE &   \\ &   \\
		-doublecouple              &   \\
		-postprocess               &   \\				 
		-partitioner               &   \\
		-postprocess               &   \\
		-problem                   &   \\   		    
		-model                     &   \\
		 &   \\
		 &   \\
		
		BOOL TYPE &   \\ &   \\	
		-top2vol-meshing           &   \\					
		-sequential                &   \\
		-timelog                   &   \\
		-useGFP                    &   \\
		-useRCM                    &   \\ 				   		
		-debug                     &   \\
		-help                      &   \\		
		
		
		
	\end{conditions*}
\end{scriptsize} 
\end{minipage}
}
\end{small}
\section{Functions in gofastplugins.cpp}

\subsection{GFPeigen}

\begin{conditions*}
GFPeigen(A,Eval,Evec); & {\ttfamily A} is a matrix, {\ttfamily Eval} is vector returning eigenvalues, {\ttfamily Evec} is the matrix returning eigenvectors.
\end{conditions*}
This is a call by reference pointer-based function of GFP library. It is used for computation of the eigenvalues and eigenvectors of a real symmetric matrix (upper triangular). This function inturn uses LAPACK libraries {\ttfamily dsyev\_} for calculation of eigenvalues and eigenvectors. 

The function {\ttfamily GFPeigen} which can be called from PSD is coded as {\ttfamily lapack\_dsyevIn} function within the gofastplugins.cpp.  this function argument 1: {\ttfamily A} is the supplied symmetric matrix, argument 2: {\ttfamily vp}  are the output eigenvalues and argument 3: {\ttfamily vectp}  are the output eigenvectors.
\begin{lstlisting}[language=PSD]
long lapack_dsyev (KNM<double> *const &A, KN<double> *const &vp, KNM<double> *const &vectp) 
{
	intblas n = A->N();
	KNM<double> mat(*A);
	
    .
    .
    dsyev_(&JOBZ, &UPLO, &n, mat, &n, *vp, w, &lw, &info);
    .
    .
    *vectp = mat;
}
\end{lstlisting}


\subsection{GFPeigenAlone}

\begin{conditions*}
GFPeigen(A,Eval,Evec); & {\ttfamily A} is a matrix, {\ttfamily Eval} is vector returning eigenvalues.
\end{conditions*}

This is a call by reference pointer based function of GFP library. It is used for computation of the eigenvalues of a real symmetric matrix (upper triangular). This function inturn uses LAPACK library for calculation of eigenvalues. The function {\ttfamily GFPeigenAlone} which can be called from PSD is coded as {\ttfamily lapack\_dsyevAlone} function within the gofastplugins.cpp. In this function argument 1: {\ttfamily A} is the supplied symmetric matrix and argument 2: {\ttfamily vp}  is the output eigenvalues of matrix {\ttfamily A} .
\begin{lstlisting}[language=PSD]
long lapack_dsyevAlone (KNM<double> *const &A, KN<double> *const &vp)
\end{lstlisting}

\subsection{GFPmaxintwoFEfields}
This is a call by reference pointer based function of GFP library. It is used to find out max between two real numbers {\ttfamily f} and {\ttfamily f1} (two 1D arrays). The max is stored in array {\ttfamily f}.
\begin{lstlisting}[language=PSD]
double GFPmaxintwoP1(KN<double> *const & f, KN<double> *const & f1)
\end{lstlisting}
