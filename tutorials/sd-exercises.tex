\newcommand{\psd}[1]{{\small\sffamily{\color{blue!60}#1}}}

\subsection{Exercise  1}

You are encouraged to try out sequential PSD solver, to do so used add
\psd{-sequential} flag to \psd{PSD\_PreProcess} step and run the solver
with \psd{PSD\_Solve\_Seq} instead of \psd{PSD\_Solve}. For example, the
PSD sequential solver workflow for the first 2D example in this tutorial
would be:

\begin{lstlisting}[style=BashInputStyle]
PSD_PreProcess -dimension 2 -problem soildynamics -dirichletconditions 1 -timediscretization newmark_beta \
-postprocess uav -sequential
\end{lstlisting}

Once the step above has been performed, we solve the problem using
\psd{PSD\_Solve\_Seq}, with the given mesh file \psd{soil.msh}.

\begin{lstlisting}[style=BashInputStyle]
PSD_Solve_Seq  Main.edp -mesh ./../Meshes/2D/soil.msh -v 0
\end{lstlisting}

Try it out for other problems of this tutorial.

\subsection{Exercise 2}

For soildynamic problems with double couple source, the double couple
source can be introduced into the solver either by displacement-based
operator -- providing displacements at the double couple points that
will be converted to moments -- or by force-based operators -- providing
forces at the double couple points that will be converted to moments. In
the tutorials above we already tried displacement-based way of
introducing double couple source by using
\psd{-doublecouple displacement\_based}. You are encouraged to try out
the force-based double couple source by using
\psd{-doublecouple force\_based}.

\subsection{Exercise 3}

You are encouraged to try out timelogging and find out if the code
(parallel/sequential) is any faster when we use Newmark-\(\beta\) or
Generalized-\(\alpha\). Read the documentation for other types of time
discretizations that can be performed with PSD, try each one out with
\psd{-timelog} and compare.

\subsection{Exercise  4}

PSD comes with additional set of plugins/functions that are highly
optimized for performing certain operations during solving. These
operations are handled by GoFast Plugins (GFP) kernel of PSD (optimize
C++ classes/templates/structures), by default this functionality is
turned off and not used. You are encouraged to try out using GFP
functions in a solver by using \psd{-useGFP} flag flag to
\psd{PSD\_PreProcess} For example, the PSD solver workflow for the first
2D example in this tutorial would be:

\begin{lstlisting}[style=BashInputStyle]
PSD_PreProcess -dimension 2 -problem soildynamics -dirichletconditions 1 -timediscretization newmark\_beta \
-postprocess uav -useGFP
\end{lstlisting}

Once the step above has been performed, we solve the problem using, with
the given mesh file \psd{soil.msh}.

\begin{lstlisting}[style=BashInputStyle]
PSD_Solve -np 4 Main.edp -mesh ./../Meshes/2D/soil.msh -v 0
\end{lstlisting}

Try it out for other problems of this tutorial. \psd{-useGFP} should
lead to a faster solver, it might be a good idea to always use this
option. To go one step further, use \psd{-timelog} flag and determine if
you have some speed up.
