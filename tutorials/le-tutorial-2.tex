\newcommand{\psd}[1]{{\small\sffamily{\color{blue!60}#1}}}

Same problem of Linear elasticity as in tutorial 1 -- 2D bar which bends
under its own load --, is discuss here. The bar \(5\times1\) m\(^2\) in
area is made up of material with \(\rho=8\times 10^3\),
\(E=200\times 10^9\), and \(\nu=0.3\). To avoid text repetition, readers
are encouraged to go ahead with this tutorial only after tutorial 1.

As the problem remains same as tutorial 1, simply add \psd{-sequential}
flag to \psd{PSD\_PreProcess} flags from tutorial 1 for sequential
solver. The flag \psd{-sequential} signifies the use of sequential PSD
solver. So the work flow for the 2D problem would be:

\begin{lstlisting}[style=BashInputStyle]
PSD_PreProcess -problem linear_elasticity -dimension 2 -bodyforceconditions 1 \
-dirichletconditions 1 -postprocess u -sequential
\end{lstlisting}

Similar to tutorial 1, We solve the problem using the given mesh file
\psd{bar.msh}. However now we need to use \psd{PSD\_Solve\_Seq} instead
of \psd{PSD\_Solve}, as such:

\begin{lstlisting}[style=BashInputStyle]
PSD_Solve_Seq Main.edp -mesh ./../Meshes/2D/bar.msh -v 0
\end{lstlisting}

Users are encouraged to try out the 3D problem with sequential solver.

\subsection{Comparing CPU time}

Naturally, since we are not using parallel PSD for solving, we lose the
advantage of solving fast. To testify this claim checking solver timings
can be helpful. PSD provides mean to time log your solver via
\psd{-timelog} flag. What this will do when you run your solver, on the
terminal you will have information printed on what is the amount of time
taken by each step of your solver. Warning, this will make your solver
slower, as this action involves `MPI\_Barrier' routines for correctly
timing operation.

An example work flow of 2D solver (parallel) with timelogging:

\begin{lstlisting}[style=BashInputStyle]
PSD_PreProcess -problem linear_elasticity -dimension 2 -bodyforceconditions 1 \
-dirichletconditions 1 -postprocess u -timelog
\end{lstlisting}

We solve the problem using four MPI processes, with the given mesh file
\psd{bar.msh}.

\begin{lstlisting}[style=BashInputStyle]
PSD_Solve -np 4 Main.edp -mesh ./../Meshes/2D/bar.msh -v 0
\end{lstlisting}

\begin{figure}[h!]
\centering
\includegraphics[width=0.4\textwidth]{./Images/le-time-par.png}
\caption{Time logging output produced for parallel run on 4 processes.\label{time-par-le}}
\end{figure}

The figure\textasciitilde{}\ref{time-par-le} shows the time logging
output produced for parallel run on 4 processes using \psd{-timelog}
flag. Take note of timings produced for different operations of the
solver.

Now let us repeat the procedure but this time use sequential solver:

\begin{lstlisting}[style=BashInputStyle]
PSD_PreProcess -problem linear_elasticity -dimension 2 -bodyforceconditions 1 \
-dirichletconditions 1 -postprocess u -timelog -sequential
\end{lstlisting}

We solve the problem now in sequential, with the given mesh file
\psd{bar.msh}.

\begin{lstlisting}[style=BashInputStyle]
PSD_Solve_Seq Main.edp -mesh ./../Meshes/2D/bar.msh -v 0
\end{lstlisting}

You should now see timings that are higher in comparison to the parallel
solver. Approximately, for large meshes using 4 MPI processes should
lead to 4 times fast solver.
