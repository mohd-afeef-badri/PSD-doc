\newcommand{\psd}[1]{{\small\sffamily{\color{blue!60}#1}}}


\subsection{Exercise 1}

You are encouraged to try out timelogging and find out if the code
(parallel/sequential) is any faster when we use Newmark-\(\beta\) or
Generalized-\(\alpha\). Read the documentation for other types of time
discretizations that can be performed with PSD, try each one out with
\psd{-timelog} and compare.

\subsection{Exercise  2}

There is a solver run level flag for mesh refinement
\footnote{Mesh refinement is performed after partitioning.}. This flag
is called \psd{-split [int]} which splits the triangles (resp.
tetrahedrons) of your mesh into four smaller triangles (resp.
tetrahedrons). As such \psd{-split 2} will produce a mesh with 4 times
the elements of the input mesh. Similarly, \psd{-split n} where \(n\) is
a positive integer produces \(2^n\) times more elements than the input
mesh. You are encouraged to use this \psd{-split} flag to produce
refined meshes and check, mesh convergence of a problem, computational
time, etc. Use of parallel computing is recommended. You could try it
out with \psd{PSD\_Solve} or \psd{PSD\_Solve\_Seq}, for example:

\begin{lstlisting}[style=BashInputStyle]
PSD_Solve -np 2 Main.edp -mesh ./../Meshes/2D/bar-dynamic.msh -v 0 -split 2
\end{lstlisting}

for splitting each triangle of the mesh \psd{bar-dynamic.msh} into 4.

\subsection{Exercise  3}

There is a preprocess level flag \psd{-debug}, which as the name
suggests should be used for debug proposes by developers. However, this
flag will activate OpebGL live plotting of the problem, with displaced
mesh. You are encouraged to try it out

\begin{lstlisting}[style=BashInputStyle]
PSD_PreProcess -dimension 2 -problem elastodynamics -dirichletconditions 1 -tractionconditions 1 \
-timediscretization newmark_beta -postprocess uav -timelog -debug
\end{lstlisting}

Then to run the problem we need aditional \psd{-wg} flag

\begin{lstlisting}[style=BashInputStyle]
PSD_Solve -np 2 Main.edp -mesh ./../Meshes/2D/bar-dynamic.msh -v 0 -wg
\end{lstlisting}

\subsection{Exercise  4}

PSD comes with additional set of plugins/functions that are highly
optimized for performing certain operations during solving. These
operations are handled by GoFast Plugins (GFP) kernel of PSD (optimize
C++ classes/templates/structures), by default this functionality is
turned off and not used. You are encouraged to try out using GFP
functions in a solver by using \psd{-useGFP} flag flag to
\psd{PSD\_PreProcess} For example, the PSD solver workflow for the first
2D example in this tutorial would be:

\begin{lstlisting}[style=BashInputStyle]
PSD_PreProcess -dimension 2 -problem elastodynamics -dirichletconditions 1 -tractionconditions 1 \
-timediscretization newmark_beta -postprocess uav -useGFP
\end{lstlisting}

Once the step above has been performed, we solve the problem using, with
the given mesh file \psd{bar-dynamic}.

\begin{lstlisting}[style=BashInputStyle]
PSD_Solve -np 2 Main.edp -mesh ./../Meshes/2D/bar-dynamic.msh -v 0 -wg
\end{lstlisting}

Try it out for other problems of this tutorial. \psd{-useGFP} should
lead to a faster solver, it might be a good idea to always use this
option. To go one step further, use \psd{-timelog} flag and determine if
you have some speed up.
