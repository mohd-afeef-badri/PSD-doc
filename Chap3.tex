\chapter{Theoretical background}

\section{Elastostatics}

Let us consider $d$-dimensional domain $\Omega\in\mathbb{R}^{d}$ in a Euclidean referential $R(O,\be_{i})$ (with $i=1,\dots,d)$ submitted to a system of body forces $\bb$. We denote $\partial\Omega$ the boundary of $\Omega$ and indicate with $\bn = \bn(\bx) = n_{i}(\bx)\be_{i}$
its outer normal in any point $\bx=x_{i}\be_{i}\in\partial\Omega$.

The problem to solve in order to characterize the dynamics equilibrium thus consists in finding a vector-valued displacement field $\bu=\bu(\bx,t):\Omega\times[0,T]\to\mathbb{R}^{d}$
regular ``enough'' and such that: 
\begin{equation}
\begin{cases}
\mathrm{\mathrm{div}}\bsigma+\bb=0 & (\bx,t)\in\Omega\times[0,T]\\
\bsigma=\bsigma(\bu) & (\bx,t)\in\Omega\times[0,T]\\
\bu=\bu^{*} & (\bx,t)\in\partial_{u}\Omega\times[0,T]\\
\bsigma\cdot\bn=\bt & (\bx,t)\in\partial_{t}\Omega\times[0,T]
\end{cases}
\end{equation}
where ``${\text{div}}$'' denotes the divergence operator, symbol ``$\cdot$'' denotes the single contraction operation between tensors, $\rho=\rho(\bx):\Omega\to\mathbb{R}$ is the material density and $\bsigma=\bsigma(\bu)$ denotes a constitutive equation
expressing the relationship between the second order Cauchy's stress tensor $\bsigma:\Omega\times[0,T]\to\mathbb{R}^{d\times d}$ and the displacement. Moreover, $\bu^{*}=\bu^{*}(\bx,t):\partial_{u}\Omega\times[0,T]\to\mathbb{R}^{d}$ is the imposed displacement field on $\partial_{u}\Omega$ (Dirichlet
boundary condition) and 
$\bt=\bt(\bx,t):\partial_{u}\Omega\times[0,T]\to\mathbb{R}^{d}$
is the imposed traction vector on $\partial_{t}\Omega$ (Neumann boundary condition). The split of $\partial\Omega$ is such that $\partial\Omega=\overline{\partial_{u}\Omega\cup\partial_{t}\Omega}$ and $\partial_{u}\Omega\cap\partial_{t}\Omega=\emptyset$, with overline $\overline{\bullet}$ denoting the closure of set $\bullet$.

Let us now introduce the spaces of the admissible displacements fields ($\mathcal{U}$) and test functions ($\mathcal{V}$):
%
\begin{equation}
\begin{aligned}\mathcal{U} & =\left\{ \bu=\bu(\bx,t):\partial_{u}\Omega\times[0,T]\to\mathbb{R}^{d}\,|\,\bu\in H^{1}(\Omega),\,\bu=\bu^{*}\,\partial_{u}\Omega\times[0,T]\right\} \\
\mathcal{V} & =\left\{ \bv=\bv(\bx,t):\partial_{u}\Omega\times[0,T]\to\mathbb{R}^{d}\,|\,\bv\in H^{1}(\Omega),\,\bv=0\,\partial_{u}\Omega\times[0,T],\right\} 
\end{aligned}
\end{equation}
%
The weak form of previous boundary value problem can be easily obtained
by integrating by part the linear momentum balance equation using
a test function $\bv\in\mathcal{V}$, and imposing the Neumann boundary
condition:
%
\begin{equation}
%
\underbrace{\int_{\Omega}\bsigma(\bu):\bepsilon(\bv)\,{\rm d}V}_{:=K(\bu,\bv)}=\underbrace{\int_{\Omega}\bb\cdot\bv\,{\rm d}V+\int_{\partial_{t}\Omega}\bt\cdot\bv\,{\rm d}S}_{:=b(\bv;\bb)+b(\bv;\bt)_{\partial_{t}\Omega}}\quad\forall\,\bv\in\mathcal{V}
%
\end{equation}

where symbol ``$:$'' is the double contraction operation between
tensors, $K(\bu,\bv)$
is the bi-linear symmetric form associated with the stiffness matrix
and $b(\bv;\bb)+b(\bv;\bt)_{\partial_{t}\Omega}$ are the linear forms
associated with the external loading.\footnote{In the following of this document, given a known field $\ba$, symbol
$b(\bv;\ba)$ will be used to denote the linear form $\int_{\Omega}\ba\cdot\bv\,{\rm d}V$,
whereas $b(\bv;\ba)_{Surf}$ will denote the linear form obtained
form the surface integral $\int_{Surf}\ba\cdot\bv\,{\rm d}S$. Any
linear form without down-script has to be interpreted as an integral
over $\Omega$. Only surface integrals will be defined
explicitly.}

The problem to solve can be finally written as:
\begin{equation}
\boxed{\begin{aligned} & \text{Find}~{\bu}~{\in}~{\mathcal{U}}~~\text{such that : }\\
 & K(\bu,\bv)=b(\bv;\bb)+b(\bv;\bt)_{\partial_{t}\Omega}\quad\forall\,\bv\in\mathcal{V}
\end{aligned}
}
\end{equation}

\section{Elastodynamics}

The problem to solve in order to characterize the dynamics equilibrium thus consists in finding a vector-valued displacement field $\bu=\bu(\bx,t):\Omega\times[0,T]\to\mathbb{R}^{d}$
regular ``enough'' and such that: 
\begin{equation}
\begin{cases}
\mathrm{\mathrm{div}}\bsigma+\bb=\rho\ddot{\bu} & (\bx,t)\in\Omega\times[0,T]\\
\bsigma=\bsigma(\bu) & (\bx,t)\in\Omega\times[0,T]\\
\bu=\bu^{*} & (\bx,t)\in\partial_{u}\Omega\times[0,T]\\
\bsigma\cdot\bn=\bt & (\bx,t)\in\partial_{t}\Omega\times[0,T]\\
\bu=\bu_{0} & \bx\in\Omega,\,t=0\\
\dot{\bu}=\dot{\bu}_{0} & \bx\in\Omega,\,t=0
\end{cases}
\end{equation}
where ``${\text{div}}$'' denotes the divergence operator, symbol ``$\cdot$'' denotes the single contraction operation between tensors, $\rho=\rho(\bx):\Omega\to\mathbb{R}$ is the material density,
$\ddot{\bu}=\ddot{\bu}(\bx,t)=\bu_{tt}:\Omega\times[0,T]\to\mathbb{R}^{d}$ is the acceleration field (i.e., the second time derivative of the field $\bu$) and $\bsigma=\bsigma(\bu)$ denotes a constitutive equation
expressing the relationship between the second order Cauchy's stress tensor $\bsigma:\Omega\times[0,T]\to\mathbb{R}^{d\times d}$ and the displacement. Moreover, $\bu^{*}=\bu^{*}(\bx,t):\partial_{u}\Omega\times[0,T]\to\mathbb{R}^{d}$ is the imposed displacement field on $\partial_{u}\Omega$ (Dirichlet
boundary condition) and 
$\bt=\bt(\bx,t):\partial_{u}\Omega\times[0,T]\to\mathbb{R}^{d}$
is the imposed traction vector on $\partial_{t}\Omega$ (Neumann boundary condition). The split of $\partial\Omega$ is such that $\partial\Omega=\overline{\partial_{u}\Omega\cup\partial_{t}\Omega}$ and $\partial_{u}\Omega\cap\partial_{t}\Omega=\emptyset$, with overline $\overline{\bullet}$ denoting the closure of set $\bullet$. Finally,
$\bu_{0}=\bu_{0}(\bx,0):\Omega\to\mathbb{R}^{d}$ and $\dot{\bu}_{0}=\dot{\bu}_{0}(\bx,0):\Omega\to\mathbb{R}^{d}$
are the displacement and velocity fields at time $t=0$ (initial conditions).

Let us now introduce the spaces of the admissible displacements fields ($\mathcal{U}$) and test functions ($\mathcal{V}$):
%
\begin{equation}
\begin{aligned}\mathcal{U} & =\left\{ \bu=\bu(\bx,t):\partial_{u}\Omega\times[0,T]\to\mathbb{R}^{d}\,|\,\bu\in H^{1}(\Omega),\,\bu=\bu^{*}\,\partial_{u}\Omega\times[0,T],\,\bu(\bx,0)=0,\:\dot{\bu(\bx,0)}=\dot{\bu}_{0}\right\} \\
\mathcal{V} & =\left\{ \bv=\bv(\bx,t):\partial_{u}\Omega\times[0,T]\to\mathbb{R}^{d}\,|\,\bv\in H^{1}(\Omega),\,\bv=0\,\partial_{u}\Omega\times[0,T],\right\} 
\end{aligned}
\end{equation}
%
The weak form of previous boundary value problem can be easily obtained
by integrating by part the linear momentum balance equation using
a test function $\bv\in\mathcal{V}$, and imposing the Neumann boundary
condition:
%
\begin{equation}
%
\underbrace{\int_{\Omega}\rho\ddot{\bu}\cdot\bv\,{\rm d}V}_{:=M(\ddot{\bu},\bv)}+\underbrace{\int_{\Omega}\bsigma(\bu):\bepsilon(\bv)\,{\rm d}V}_{:=K(\bu,\bv)}=\underbrace{\int_{\Omega}\bb\cdot\bv\,{\rm d}V+\int_{\partial_{t}\Omega}\bt\cdot\bv\,{\rm d}S}_{:=b(\bv;\bb)+b(\bv;\bt)_{\partial_{t}\Omega}}\quad\forall\,\bv\in\mathcal{V}
%
\end{equation}

where symbol ``$:$'' is the double contraction operation between
tensors, $M(\ddot{\bu},\bv)$ is the bi-linear symmetric form associated
with inertial terms (i.e., dependent on the mass matrix), $K(\bu,\bv)$
is the bi-linear symmetric form associated with the stiffness matrix
and $b(\bv;\bb)+b(\bv;\bt)_{\partial_{t}\Omega}$ are the linear forms
associated with the external loading.\footnote{In the following of this document, given a known field $\ba$, symbol
$b(\bv;\ba)$ will be used to denote the linear form $\int_{\Omega}\ba\cdot\bv\,{\rm d}V$,
whereas $b(\bv;\ba)_{Surf}$ will denote the linear form obtained
form the surface integral $\int_{Surf}\ba\cdot\bv\,{\rm d}S$. Any
linear form without down-script has to be interpreted as an integral
over $\Omega$. Only surface integrals will be defined
explicitly.}

The problem to solve can be finally written as:
\begin{equation}
\boxed{\begin{aligned} & \text{Find}~~{\bu}~{\in}~{\mathcal{U}}~~\text{such that : }\\
 & M(\ddot{\bu},\bv)+K(\bu,\bv)=b(\bv;\bb)+b(\bv;\bt)_{\partial_{t}\Omega}\quad\forall\,\bv\in\mathcal{V}
\end{aligned}
}
\end{equation}

The only way for accounting form dumping effects in this formulation
is through a proper definition of a constitutive law $\bsigma=\bsigma(\bu)$
modeling dissipative processes occurring at the material level. In
some cases, however, it can be useful to account for damping effects
in a more global way. This can be done by modifying the variational
problem as follows:
\begin{equation}
\boxed{\begin{aligned} & \text{Find}~{\bu}~{\in}~{\mathcal{U}}~~\text{such that : }\\
 & M(\ddot{\bu},\bv)+C(\dot{\bu},\bv)+K(\bu,\bv)=b(\bv;\bb)+b(\bv;\bt)_{\partial_{t}\Omega}\quad\forall\,\bv\in\mathcal{V}
\end{aligned}
}
\end{equation}
where $C(\dot{\bu},\bv)$ is an additional bi-linear symmetric form
associated with damping/viscous effects.

\section{Time discretization}

Time discretized variational formulations are illustrate in this subsection,
considering several implicit time integration schemes. Representative
members of these algorithms are, among others, the N\textminus $\beta$
method \cite{newmark1959method}, the HHT\textminus $\alpha$ method \cite{hilber1977improved}, the WBZ\textminus $\alpha$ method \cite{wood1980alpha}, the HP\textminus $\theta_{1}$ method \cite{hoff1988development} and the CH\textminus $\alpha$ method \cite{chung1993time}. These methods exhibit second order accuracy
in linear dynamics and permit efficient variable step size techniques,
being one-step methods. The CH\textminus $\alpha$, the HHT\textminus $\alpha$
and the WBZ\textminus \textgreek{a} methods, the so called $\alpha-$methods,
are one-parameter schemes which can be considered as particular cases
of a more general class of methods named generalized \textminus{}
$\alpha$ (G \textminus{} $\alpha$). This class
of methods corresponds to the CH\textminus \textgreek{a} scheme \cite{chung1993time}, where the algorithmic parameters $\alpha_{m}$,
$\alpha_{f}$, $\beta$ and $\gamma$ are assumed to be independent
of each other.

\subsection{Generalized-\texorpdfstring{$\alpha$}{a} method}

The Generalized \textminus{} $\alpha$ (G \textminus{} $\alpha$)
is an implicit method that allows for high frequency energy dissipation,
reduced unwanted low-frequency dissipation, and second order accuracy
(i.e., $\Delta t^{2}$), both in linear and nonlinear regimes. Depending
on choices of input parameters, unconditionally stability can be achieved
for linear problems (as for all implicit schemes). Stability properties
for nonlinear problem were studied in \cite{erlicher2002analysis}. In
the latter work, the second-order accuracy of this class of algorithms
was proved also in the non-linear regime, independently of the quadrature
rule for non-linear internal forces. Conversely, the G-stability notion
which is suitable for linear multi-step schemes devoted to non-linear
dynamic problems cannot be applied, as the non-linear structural dynamics
equations are not contractive. Nonetheless, \cite{erlicher2002analysis}
proved that the G \textminus{} $\alpha$ methods are stable in an
energy sense, and guarantee energy decay for high-frequencies and
asymptotic cancellation. However, overshoot and heavy energy oscillations
in the intermediate-frequency range are exhibited.

\subsubsection{Problem setting}

Let introduce a time discretization of the time interval $[0,T]$
in an ordered sequence of $N+1$ time increments $(0,\dots,t_{i},t_{i+1},\dots,T$)
such that $t_{i+1}=t_{i}+\Delta t$, with $\Delta t=T/N$ denoting
the time step (here supposed constant). According to the (G \textminus{}
$\alpha$) method, the dynamic evolution equation is solved at intermediate
time $t_{n+1-\alpha}\in[t_{n},t_{n+1}]$. The following notation is
used to denote the value of a generic variable $z$ at time $t_{n+1-\alpha}$:
\begin{equation}
z_{n+1\text{\textminus}\alpha}=(1\text{\textminus}\alpha)z_{n+1}+\alpha z_{n}\quad \text{with} \quad \alpha \in [0,1]\label{eq:Xn1alpha}
\end{equation}
Furthermore, the following approximations (standard for Newmark schemes)
for the displacement and velocity fields at time $t_{n+1}$ are used \cite{newmark1959method}:
\begin{equation}
\begin{aligned}{\bu_{n+1}} & =\overline{\bu}_{n+1}+\beta\,\Delta t^{2}\,\ddot{\bu}_{n+1}\\
\dot{\bu}_{n+1} & =\dot{\overline{\bu}}_{n+1}+\gamma\,\Delta t\,\ddot{\bu}_{n+1}
\end{aligned}
\label{eq:newmmark_approximations-corr-1}
\end{equation}
where $\overline{\bu}_{n+1}$ and $\dot{\overline{\bu}}_{n+1}$ are
the following known contributions (predictions in predictor-corrector
schemes):
\begin{equation}
\begin{aligned}\overline{\bu}_{n+1} & ={\bu_{n}}+\Delta t\,\dot{\bu}_{n}+\Delta t^{2}\left(\frac{1}{2}\text{\textminus}\beta\right)\ddot{\bu}_{n}\\
\dot{\overline{\bu}}_{n+1} & =\dot{\bu}_{n}+\Delta t\,(1\text{\textminus}\gamma)\ddot{\bu}_{n}
\end{aligned}
\label{eq:newmmark_approximations-pred-1}
\end{equation}

and $(\beta,\gamma)$ are algorithmic parameters.
By inverting the
first equation of (\ref{eq:newmmark_approximations-corr-1}), one
can express $\ddot{\bu}_{n+1}$ as a function of $\bu_{n+1}$ as:
\begin{equation}
\ddot{\bu}_{n+1}=\frac{1}{\beta\,\Delta t^{2}}\left(\bu_{n+1}-\overline{\bu}_{n+1}\right)\label{eq:accn1fdispn1}
\end{equation}


\subsection{Time discretized variational problem (no damping)}

Neglecting damping effects, the problem to solve is written as:
\begin{equation}
\boxed{\begin{aligned} & \text{Find}~{\bu_{n+1}~{\in}~{\mathcal{U}}}~~\text{such that : }\\
 & M(\ddot{\bu}_{n+1-\alpha_{m}},\bv)+K(\bu_{n+1-\alpha_{f}},\bv)=b(\bv;\bb)+b(\bv;\bt_{n+1-\alpha_{f}})_{\partial_{t}\Omega}\quad{{\forall\,}}\bv\in\mathcal{V}
\end{aligned}
}
\end{equation}
where $\ddot{\bu}_{n+1-\alpha_{m}}$ and $\bu_{n+1-\alpha_{f}}$ can
be written according to (\ref{eq:Xn1alpha}):
\begin{equation}
\begin{aligned}\ddot{\bu}_{n+1-\alpha_{m}} & =\frac{1-\alpha_{m}}{\beta\,\Delta t^{2}}\left(\bu_{n+1}-\overline{\bu}_{n+1}\right)+\alpha_{m}\ddot{\bu}_{n}\\
\bu_{n+1-\alpha_{f}} & =(1-\alpha_{f})\bu_{n+1}+\alpha_{f}\bu_{n}
\end{aligned}
\label{eq:accn1fdispn1-1}
\end{equation}
Furthermore, parameters $\beta$ and $\gamma$ read:
\begin{equation}
    \gamma = \frac{1}{2}+ \alpha_f - \alpha_m \qquad \beta = \frac{1}{4} \left( \gamma + \frac{1}{2}\right)^2
\end{equation}

\paragraph{Bilinear and linear operators.}

Using equation (\ref{eq:accn1fdispn1}), one can easily write the
bi-linear part associated with the mass matrix in terms of the unknown
displacement $\bu_{n+1}$ as follows:
\begin{equation}
\begin{aligned}M(\ddot{\bu}_{n+1-\alpha_{m}},\bv) & =\frac{1-\alpha_{m}}{\beta\,\Delta t^{2}}M(\bu_{n+1},\bv)-\frac{1-\alpha_{m}}{\beta\,\Delta t^{2}}m(\bv;\overline{\bu}_{n+1})+\alpha_{m}m(\bv;\ddot{\bu}_{n})\end{aligned}
\label{eq:M_1}
\end{equation}
where linear forms $m(\bv;\overline{\bu}_{n+1})$ and $m(\bv;\ddot{\bu}_{n})$
read:\footnote{More in general, given a field $\ba=\ba(\bx):\Omega\to\mathbb{R}^{d}$,
$m(\bv;\ba)$ denotes the linear form:
\begin{equation}
m(\bv;\ba)=\int_{\Omega}\rho\ba\cdot\bv\,{\rm d}V
\end{equation}}
\begin{equation}
m(\bv;\overline{\bu}_{n+1})=\int_{\Omega}\rho\overline{\bu}_{n+1}\cdot\bv\,{\rm d}V\qquad m(\bv;\ddot{\bu}_{n})=\int_{\Omega}\rho\ddot{\bu}_{n}\cdot\bv\,{\rm d}V
\end{equation}

Term $m(\bv;\overline{\bu}_{n+1})$ figuring in equation (\ref{eq:M_1})
can also be expanded as:
\begin{equation}
m(\bv;\overline{\bu}_{n+1})=m(\bv;{\bu_{n}})+\Delta t\,m(\bv;\dot{\bu}_{n})+\Delta t^{2}\left(\frac{1}{2}\text{\textminus}\beta\right)m(\bv;\ddot{\bu}_{n})
\end{equation}
As a consequence (\ref{eq:M_1}) can be rewritten as:\footnote{When summing up the terms depending on $\ddot{\bu}_{n}$, coming from
the definition of $\overline{\bu}_{n+1}$ and from equation (\ref{eq:M_1}),
we have:
\begin{equation}
-\left[(1-\alpha_{m})\left(\frac{1-2\beta}{2\beta}\right)-\alpha_{m}\right]=-\frac{(1-\alpha_{m})(1-2\beta)-2\beta\alpha_{m}}{2\beta}=-\frac{1-2\beta-\alpha_{m}+2\beta\alpha_{m}-2\beta\alpha_{m}}{2\beta}=1-\frac{1-\alpha_{m}}{2\beta}
\end{equation}
}
\begin{equation}
M(\ddot{\bu}_{n+1-\alpha_{m}},\bv)=\frac{1-\alpha_{m}}{\beta\,\Delta t^{2}}M(\bu_{n+1},\bv)-\frac{1-\alpha_{m}}{\beta\,\Delta t^{2}}m(\bv;\bu_{n})-\frac{1-\alpha_{m}}{\beta\,\Delta t}m(\bv;\dot{\bu}_{n})+\left(1-\frac{1-\alpha_{m}}{2\beta}\right)m(\bv;\ddot{\bu}_{n})
\end{equation}

In a similar way, we can rewrite the bi-linear form associated with
the stiffness matrix as:
\begin{equation}
\begin{aligned}K(\bu_{n+1-\alpha_{f}},\bv) & =(1-\alpha_{f})K(\bu_{n+1},\bv)+\alpha_{f}k(\bv;\bu_{n})\end{aligned}
\end{equation}

where $k(\bv;\bu_{n})$ is the linear form:\footnote{More in general, given a field $\ba=\ba(\bx):\Omega\to\mathbb{R}^{d}$,
$k(\bv;\ba)$ denotes the linear form:
\begin{equation}
k(\bv;\ba)=\int_{\Omega}\bsigma\left(\ba\right):\bepsilon(\bv)\,{\rm d}V
\end{equation}}
\begin{equation}
k(\bv;\bu_{n})=\int_{\Omega}\bsigma\left(\bu_{n}\right):\bepsilon(\bv)\,{\rm d}V
\end{equation}
Finally, the linear form $b(\bv;\bt_{n+1-\alpha_{f}})_{\partial_{t}\Omega}$
becomes:
\begin{equation}
b(\bv;\bt_{n+1-\alpha_{f}})_{\partial_{t}\Omega}=(1-\alpha_{f})b(\bv;\bt_{n+1})_{\partial_{t}\Omega}+\alpha_{f}b(\bv;\bt_{n})_{\partial_{t}\Omega}
\end{equation}


\paragraph{Final variational problem.}

The time discretized variational formulation to solve becomes:
\begin{equation}
\boxed{\begin{aligned} & \text{Find}~{\bu_{n+1}~{\in}~{\mathcal{U}}}~~\text{such that : }\\
 & \widetilde{K}({\bu_{n+1}},\bv)=\widetilde{l}(\bv)
\end{aligned}
}
\end{equation}
where $ \widetilde{K}({\bu_{n+1}},\bv)$ is the bi-linear form associated with
the effective/equivalent stiffness matrix:
\begin{equation}
\begin{aligned}\widetilde{K}({\bu_{n+1}},\bv) & =\frac{1-\alpha_{m}}{\beta\,\Delta t^{2}}M(\bu_{n+1},\bv)+(1-\alpha_{f})K(\bu_{n+1},\bv)\end{aligned}
\end{equation}
and $\widetilde{l}(\bv)=\widetilde{l}(\bv;\left\{ \bb,\bt_{n},\bt_{n+1},\bu_{n},\dot{\bu}_{n},\ddot{\bu}_{n}\right\} )$
is the following linear form:
\begin{equation}
\begin{aligned}\widetilde{l}(\bv) & =b(\bv;\bb)+b(\bv;\bt_{n+1-\alpha_{f}})_{\partial_{t}\Omega}+\frac{1-\alpha_{m}}{\beta\,\Delta t^{2}}m(\bv,\bu_{n})+\frac{1-\alpha_{m}}{\beta\,\Delta t}m(\bv;\dot{\bu}_{n})\\
 & \qquad\cdots+\left(1-\frac{1-\alpha_{m}}{2\beta}\right)m(\bv;\ddot{\bu}_{n})-\alpha_{f}k(\bv;{\bu_{n}})
\end{aligned}
\end{equation}


\subsection{Time discretized variational problem (Rayleigh damping)}

The problem to solve is now:
\begin{equation}
\boxed{\begin{aligned} & \text{Find}~{\bu_{n+1}~{\in}~{\mathcal{U}}}~~\text{such that : }\\
 & M(\ddot{\bu}_{n+1-\alpha_{m}},\bv)+C(\dot{\bu}_{n+1-\alpha_{f}},\bv)+K(\bu_{n+1-\alpha_{f}},\bv)=b(\bv;\bb)+b(\bv;\bt_{n+1-\alpha_{f}})_{\partial\Omega}\quad{{\forall\,}}\bv\in\mathcal{V}
\end{aligned}
}\label{eq:variational_pb_damping_1}
\end{equation}
where, following a simple Rayleigh formulation, the bi-linear form associated with the damping matrix can be written as:
\begin{equation}
C(\dot{\bu}_{n+1-\alpha_{f}},\bv)=\eta_{M}M(\dot{\bu}_{n+1-\alpha_{f}},\bv)+\eta_{K}K(\dot{\bu}_{n+1-\alpha_{f}},\bv)
\end{equation}
with ($\eta_{M},\eta_{K}$) denoting two positive model parameters.

Now, using definitions \eqref{eq:Xn1alpha}, \eqref{eq:newmmark_approximations-corr-1}
and \eqref{eq:newmmark_approximations-pred-1}, $\dot{\bu}_{n+1-\alpha_{f}}$ can be written as:\footnote{Using definitions \eqref{eq:Xn1alpha}, \eqref{eq:newmmark_approximations-corr-1}
and \eqref{eq:newmmark_approximations-pred-1}, the velocity field at time $t_{n+1-\alpha_{f}}$ reads:
\begin{equation}
\begin{aligned}\dot{\bu}_{n+1-\alpha_{f}} & =(1-\alpha_{f})\dot{\bu}_{n+1}+\alpha_{f}\dot{\bu}_{n}\\
 & =(1-\alpha_{f})\dot{\overline{\bu}}_{n+1}+\alpha_{f}\dot{\bu}_{n}+\gamma\,\Delta t\,(1-\alpha_{f})\ddot{\bu}_{n+1}\\
 & =\frac{\gamma\,(1-\alpha_{f})}{\beta\,\Delta t}\bu_{n+1}+(1-\alpha_{f})\dot{\overline{\bu}}_{n+1}-\frac{\gamma(1-\alpha_{f})}{\beta\,\Delta t}\overline{\bu}_{n+1}+\alpha_{f}\dot{\bu}_{n}
\end{aligned}
\end{equation}
}
\begin{equation}
\begin{aligned}\dot{\bu}_{n+1-\alpha_{f}} & =\frac{\gamma\,(1-\alpha_{f})}{\beta\,\Delta t}\bu_{n+1}+(1-\alpha_{f})\dot{\overline{\bu}}_{n+1}-\frac{\gamma(1-\alpha_{f})}{\beta\,\Delta t}\overline{\bu}_{n+1}+\alpha_{f}\dot{\bu}_{n}\end{aligned}
\end{equation}
or, using definitions (\ref{eq:newmmark_approximations-pred-1}),
as:\footnote{Using definitions (\ref{eq:newmmark_approximations-pred-1}) one obtains:
\begin{equation}
\begin{aligned}\dot{\bu}_{n+1-\alpha_{f}} & =\frac{\gamma\,(1-\alpha_{f})}{\beta\,\Delta t}\bu_{n+1}+(1-\alpha_{f})\left[\dot{\bu}_{n}+\Delta t\,(1\text{\textminus}\gamma)\ddot{\bu}_{n}\right]-\frac{\gamma(1-\alpha_{f})}{\beta\,\Delta t}\left[{\bu_{n}}+\Delta t\,\dot{\bu}_{n}+\Delta t^{2}\left(\frac{1}{2}\text{\textminus}\beta\right)\ddot{\bu}_{n}\right]+\alpha_{f}\dot{\bu}_{n}\\
 & =\frac{\gamma\,(1-\alpha_{f})}{\beta\,\Delta t}\bu_{n+1}+\left[1-\frac{\gamma(1-\alpha_{f})}{\beta}\right]\dot{\bu}_{n}+(1-\alpha_{f})\Delta t\left\{ 1\text{\textminus}\gamma\left[1+\left(\frac{1-2\beta}{2\beta}\right)\right]\right\} \ddot{\bu}_{n}-\frac{\gamma(1-\alpha_{f})}{\beta\,\Delta t}{\bu_{n}}\\
 & =\frac{\gamma\,(1-\alpha_{f})}{\beta\,\Delta t}\bu_{n+1}-\frac{\gamma(1-\alpha_{f})}{\beta\,\Delta t}{\bu_{n}}-\left[\frac{\gamma(1-\alpha_{f})}{\beta}-1\right]\dot{\bu}_{n}-\Delta t(1-\alpha_{f})\left(\frac{\gamma}{2\beta}-1\right)\ddot{\bu}_{n}
\end{aligned}
\end{equation}
}
\begin{equation}
\begin{aligned}\dot{\bu}_{n+1-\alpha_{f}} & =\frac{\gamma\,(1-\alpha_{f})}{\beta\,\Delta t}\bu_{n+1}-\frac{\gamma(1-\alpha_{f})}{\beta\,\Delta t}{\bu_{n}}-\left[\frac{\gamma(1-\alpha_{f})}{\beta}-1\right]\dot{\bu}_{n}-\Delta t(1-\alpha_{f})\left(\frac{\gamma}{2\beta}-1\right)\ddot{\bu}_{n}\end{aligned}
\end{equation}


\paragraph{Bilinear and linear operators.}

Operator $M(\dot{\bu}_{n+1-\alpha_{f}},\bv)$ reads:
\begin{equation}
\begin{aligned}M(\dot{\bu}_{n+1-\alpha_{f}},\bv) & =\frac{\gamma\,(1-\alpha_{f})}{\beta\,\Delta t}M(\bu_{n+1},\bv)-\frac{\gamma(1-\alpha_{f})}{\beta\,\Delta t}m(\bv;{\bu_{n}})\\
 & \quad\cdots-\left[\frac{\gamma(1-\alpha_{f})}{\beta}-1\right]m(\bv;\dot{\bu}_{n})-\Delta t(1-\alpha_{f})\left(\frac{\gamma}{2\beta}-1\right)m(\bv;\ddot{\bu}_{n})
\end{aligned}
\end{equation}
Similarly, the stiffness contribution becomes:
\begin{equation}
\begin{aligned}K(\dot{\bu}_{n+1-\alpha_{f}},\bv) & =\frac{\gamma\,(1-\alpha_{f})}{\beta\,\Delta t}K(\bu_{n+1},\bv)-\frac{\gamma\,(1-\alpha_{f})}{\beta\,\Delta t}k(\bv;\bu_{n})\\
 & \quad\cdots-\left[\frac{\gamma(1-\alpha_{f})}{\beta}-1\right]k(\bv;\dot{\bu}_{n})-\Delta t(1-\alpha_{f})\left(\frac{\gamma}{2\beta}-1\right)k(\bv;\ddot{\bu}_{n})
\end{aligned}
\end{equation}


\paragraph{Final variational problem.}

Finally, the variational problem to solve reads:
\begin{equation}
\boxed{\begin{aligned} & \text{Find}~{\bu_{n+1}~{\in}~{\mathcal{U}}}~~\text{such that : }\\
 & \widetilde{\widetilde{K}}({\bu_{n+1}},\bv)=\widetilde{\widetilde{l}}(\bv)\quad{{\forall\,}}\bv\in\mathcal{V}
\end{aligned}
}
\end{equation}
where $\widetilde{\widetilde{K}}({\bu_{n+1}},\bv)$ is the bi-linear form associated with the effective stiffness matrix:
\begin{equation}
\begin{aligned}\widetilde{\widetilde{K}}({\bu_{n+1}},\bv) & =\widetilde{K}({\bu_{n+1}},\bv)+\frac{\gamma\,(1-\alpha_{f})}{\beta\,\Delta t}C({\bu_{n+1}},\bv)\\
 & =\frac{1-\alpha_{m}}{\beta\Delta t^{2}}M({\bu_{n+1}},\bv)+\frac{\gamma\,(1-\alpha_{f})}{\beta\,\Delta t}C({\bu_{n+1}},\bv)+\left(1-\alpha_{f}\right)K({\bu_{n+1}},\bv)
\end{aligned}
\label{eq:final_varational_discr_dumping}
\end{equation}
%
with $C({\bu_{n+1}},\bv)$ denoting  the Rayleigh damping operator:
%
\begin{equation}
C({\bu_{n+1}},\bv)=\eta_{M}M({\bu_{n+1}},\bv)+\eta_{K}K({\bu_{n+1}},\bv)
\end{equation}
%
and $\widetilde{\widetilde{l}}(\bv) =\widetilde{\widetilde{l}}(\bv;\left\{ \bb,\bt_{n},\bt_{n+1},\bu_{n},\dot{\bu}_{n},\ddot{\bu}_{n}\right\} )$ being the following linear form:
%
\begin{equation}
\begin{aligned}\widetilde{\widetilde{l}}(\bv) & =\widetilde{l}(\bv)+\Delta t(1-\alpha_{f})\left(\frac{\gamma}{2\beta}-1\right)c(\bv;\ddot{\bu}_{n})\\
 & \quad\cdots+\left[\frac{\gamma(1-\alpha_{f})}{\beta}-1\right]c(\bv;\dot{\bu}_{n})+\frac{\gamma(1-\alpha_{f})}{\beta\,\Delta t}c(\bv;{\bu_{n}})
\end{aligned}
\end{equation}
In previous equation we introduced the following notation:
\begin{equation}
c(\bv;\ba)=\eta_{M}m(\bv;\ba)+\eta_{K}k(\bv;\ba)
\end{equation}

\subsection{Implicit N\textminus \texorpdfstring{$\beta$}{b} and HHT\textminus \texorpdfstring{$\alpha$}{a} method
as special cases}

\paragraph{Newmark.}

One can easily show that, the Newmark scheme is obtained by choosing
$\alpha_{m}=\alpha_{f}=0$.

Without damping, the stiffness matrix becomes:
\begin{equation}
\widetilde{K}({\bu_{n+1}},\bv)=\frac{1}{\beta\Delta t^{2}}M({\bu_{n+1}},\bv)+K({\bu_{n+1}},\bv)
\end{equation}
whereas the linear form simplifies as follows:
\begin{equation}
\begin{aligned}\widetilde{l}(\bv) & =b(\bv;\bb)+b(\bv;\bt_{n+1})_{\partial_{t}\Omega}+\frac{1}{\beta\,\Delta t^{2}}m(\bv;\overline{\bu}_{n+1})\\
 & =b(\bv;\bb)+b(\bv;\bt_{n+1})_{\partial_{t}\Omega}+\frac{1}{\beta\,\Delta t^{2}}\left[m(\bv;\bu_{n})+\Delta t\,m(\bv;\dot{\bu}_{n})+\Delta t^{2}\left(\frac{1}{2}-\beta\right)m(\bv;\ddot{\bu}_{n})\right]
\end{aligned}
\end{equation}
When Rayleigh damping is considered, the bi-linear operator $\widetilde{\widetilde{K}}({\bu_{n+1}},\bv)$
becomes:
\begin{equation}
\widetilde{\widetilde{K}}({\bu_{n+1}},\bv)=\frac{1}{\beta\Delta t^{2}}M({\bu_{n+1}},\bv)+\frac{\gamma}{\beta\,\Delta t}C({\bu_{n+1}},\bv)+K({\bu_{n+1}},\bv)
\end{equation}
whereas the linear form simplifies as follows:
\begin{equation}
\begin{aligned}\widetilde{\widetilde{l}}(\bv) & =\widetilde{l}(\bv)+\Delta t\left(\frac{\gamma}{2\beta}-1\right)c(\bv;\ddot{\bu}_{n})+\left(\frac{\gamma}{\beta}-1\right)c(\bv;\dot{\bu}_{n})+\frac{\gamma}{\beta\,\Delta t}c(\bv;{\bu_{n}})\end{aligned}
\end{equation}


\paragraph{HHT.}

One can also show that HHT\textminus $\alpha$ \cite{hilber1977improved} method is recovered for $\alpha_{m}=0$. Such formulation is not detailed in the following of this document, since it is less used than the classic Newmark approach.

\subsection{Considerations on methods based upon operator splitting}

In order to introduce predictor-correction, implicit-explicit and
more in general schemes based upon operator splitting, one can rewrite
displacement and velocity in a predictor-correction fashion as in
(\ref{eq:newmmark_approximations-corr-1}) and (\ref{eq:newmmark_approximations-pred-1}),
where (\ref{eq:newmmark_approximations-pred-1}) now defines predictors
and (\ref{eq:newmmark_approximations-corr-1}) correctors (for more general information, the interested reader can refer to \cite{hughes2012finite,zienkiewicz1994finite}). For
instance, a simple explicit predictor-corrector method can be obtained
through solving problem (\ref{eq:variational_pb_damping_1}) with
$K(\bar{\bu}_{n+1},\bv$) and $C((1-\alpha_{f})\dot{\overline{\bu}}_{n+1}+\alpha_{f}\bu_{n})$.
Mixed implicit-explicit predictor-corrector methods can also be obtained
through splitting $\Omega$ into two subdomains and using different
time-integration schemes for solving the dynamic equilibrium problem
on each of them.

\section{Space discretization}

Space discretization is performed according to the standard finite
element method. The computational domain $\Omega$ is thus discretized
into a mesh $\Omega^{h}$ comprising a finite number ($n_{el}$) of
subdomains, the finite element $\Omega_{e}^{h}$, such that $\Omega\approx\Omega^{h}=\cup_{e=1}^{n_{el}}\Omega_{e}^{h}$.
Inside each element, the displacement field is interpolated based
on nodal displacements ($\bd$) through the shape functions matrix
($\bN$), i.e., $\bu\approx\bu^{h}=\bN(\bx)\bd$. As usual, gradient
terms are interpolated using the derivatives of the shape functions,
i.e., $\bepsilon\approx\bepsilon^{h}=\bB(\bx)\bd$.

In a standard matrix format, after spatial discretization of the displacement
field, the problem to solve can be written in the standard form as:
\begin{equation}
\boxed{\begin{aligned} & \text{Find}~{\bd_{n+1}}~~\text{such that : }\\
 & \left[\frac{1-\alpha_{m}}{\beta\Delta t^{2}}\bM+\left(1-\alpha_{f}\right)\bK\right]{\bd_{n+1}}\\
 & =\bfvec_{n+1-\alpha_{f}}+\bM\left[\left(\frac{1-\alpha_{m}}{2\beta}-1\right)\ddot{\bd}_{n}+\frac{1-\alpha_{m}}{\beta\,\Delta t}\dot{\bd}_{n}+\frac{1-\alpha_{m}}{\beta\,\Delta t^{2}}{\bd_{n}}\right]-\alpha_{f}\bK{\bd_{n}}
\end{aligned}
}
\end{equation}
where $\bM$ and $\bK$ are now the mass and stiffness matrices. They
are obtained through assembling (operator $\assemble$) the corresponding
elemental operators over the finite element mesh as:
\begin{equation}
\begin{aligned}\bM & =\assemble_{e=1}^{n_{el}}\int_{\Omega_{e}}\rho\bN^{\top}\bN\mathrm{d}V\\
\bK & =\assemble_{e=1}^{n_{el}}\int_{\Omega_{e}}\bB^{\top}\bD\bB\mathrm{d}V
\end{aligned}
\end{equation}
where $\bD$ is the material stiffness matrix defining the link between
the stress and strain tensors (or between theirs rates of variation
-- more details are given in the next section). 

When Rayleigh damping is considered the problem to solve is written
as:
\begin{equation}
\boxed{\begin{aligned} & \text{Find}~{\bd_{n+1}}~~\text{such that : }\\
 & \left[\frac{1-\alpha_{m}}{\beta\Delta t^{2}}\bM+\frac{\gamma\,(1-\alpha_{f})}{\beta\,\Delta t}\bC+\left(1-\alpha_{f}\right)\bK\right]{\bd_{n+1}}\\
 & =\bfvec_{n+1-\alpha_{f}}+\bM\left[\left(\frac{1-\alpha_{m}}{2\beta}-1\right)\ddot{\bd}_{n}+\frac{1-\alpha_{m}}{\beta\,\Delta t}\dot{\bd}_{n}+\frac{1-\alpha_{m}}{\beta\,\Delta t^{2}}{\bd_{n}}\right]-\alpha_{f}\bK{\bd_{n}}\\
 & +\bC\left\{ \Delta t(1-\alpha_{f})\left(\frac{\gamma}{2\beta}-1\right)\ddot{\bd}_{n}+\left[\frac{\gamma(1-\alpha_{f})}{\beta}-1\right]\dot{\bd}_{n}+\frac{\gamma\,(1-\alpha_{f})}{\beta\,\Delta t}\bd_{n}\right\} 
\end{aligned}
}
\end{equation}

where matrix $\bC$ is now defined as:
\begin{equation}
\bC=\eta_{M}\bM+\eta_{K}\bK
\end{equation}


\section{Linear and nonlinear dynamic solvers}

Elastodynamics is the simplest case one can encounter in structural mechanics. In that case, the space-time discretized linear system of equations is linear and finding the solution at any time $t_{n+1}$ is straightforward. In most applications, however, material behavior is nonlinear since structural materials often dissipate energy and exhibit
damage, permanent strains, etc. In that case, the resulting discretized problem to solve is nonlinear, and Newton--Raphson procedures can be used. In that case, the solution is found iteratively through solving a series of linearized problems.

\subsection{Linear case - linear elastic material behavior}

Let us start from the linear case first. Under small strains conditions, if the material is assumed isotropic linear elastic, the Cauchy's stress tensor reads $\bsigma=\lambda\text{tr}(\bepsilon)\bI+2\mu\bepsilon$,
where $\bepsilon=(\nabla\bu+\nabla^{\top}\bu)/2$ is the small strain tensor (i.e., the symmetric part of the displacement gradient $\nabla\bu$), $\lambda=\lambda(\bx$) and $\mu=\mu(\bx)$ are the Lame's parameters and $\bI=\delta_{ij}\be_{i}\otimes\be_{j}$ denotes the second order identity tensor. As a consequence, the bi-linear form $K({\bu_{n+1}},\bv)$ and the corresponding linear form $k(\bv;{\bu_{n}})$ can be rewritten in a more explicit form as:
\begin{equation}
\begin{aligned}K({\bu_{n+1}},\bv)=\int_{\Omega}\bsigma\left({\bu_{n+1}}\right):\bepsilon(\bv)\,{\rm d}V & =\int_{\Omega}\left[\lambda\tr\bepsilon(\bu_{n+1})\bI+2\mu\bepsilon(\bu_{n+1})\right]:\bepsilon(\bv)\,{\rm d}V=\int_{\Omega}\bepsilon(\bu_{n+1}):\mathbb{E}:\bepsilon(\bv)\,{\rm d}V\\
k(\bv;{\bu_{n}})=\int_{\Omega}\bsigma\left({\bu_{n}}\right):\bepsilon(\bv)\,{\rm d}V & =\int_{\Omega}\left[\lambda\tr\bepsilon(\bu_{n})\bI+2\mu\bepsilon(\bu_{n})\right]:\bepsilon(\bv)\,{\rm d}V=\int_{\Omega}\bepsilon(\bu_{n}):\mathbb{E}:\bepsilon(\bv)\,{\rm d}V
\end{aligned}
\end{equation}
where $\mathbb{E}$ is the fourth-order elastic stiffness tensor. 

\subsection{Nonlinear case - inelastic material behaviors (under implementation)}

An iterative Newton--Raphson procedure is used to solve the nonlinear
problem. The unknown displacement ${\bu_{n+1}}$ at global iteration
$k+1$ is written as ${\bu_{n+1}^{k+1}}={\bu_{n+1}^{k}}+\delta{\bu_{n+1}^{k+1}}$,
where ${\bu_{n+1}^{k}}$ is the solution at iteration $k$, and $\delta{\bu_{n+1}^{k+1}}$
is the solution variation at iteration $k+1$. The latter is computed
from the resolution of a linearized system of equations. 

For this purpose, the variational formulation (\ref{eq:final_varational_discr_dumping}) is first written in residual form as:
\begin{equation}
\boxed{\begin{aligned} & \text{Find}~{{\bu_{n+1}^{k+1}}~{\in}~{\mathcal{U}}}~~\text{such that : }\\
 & R({\bu_{n+1}^{k+1}},\bv)=\widetilde{\widetilde{K}}(\bu_{n+1}^{k+1},\bv)-\widetilde{\widetilde{l}}(\bv)=0\quad{{\forall\,}}\bv\in\mathcal{V}
\end{aligned}
}
\end{equation}

The residual(i.e., the out-of-balance force) is then linearized around solution ${\bu_{n+1}^{k}}$
as follows:
\begin{equation}
R({\bu_{n+1}^{k+1}},\bv)=r(\bv;\bu_{n+1}^{k})+R'(\delta{\bu_{n+1}^{k+1}},\bv;\bu_{n+1}^{k})\label{eq:linearization_1-1}
\end{equation}
where $r(\bv;\bu_{n+1}^{k})=r(\bv;\left\{ \bb,\bt_{n},\bt_{n+1},\bu_{n},\dot{\bu}_{n},\ddot{\bu}_{n}\right\} ,\bu_{n+1}^{k})$
is the linear form corresponding to the out-of-balance forces at iteration
$k$:
\begin{equation}
r(\bv;\bu_{n+1}^{k})=\frac{1-\alpha_{m}}{\beta\Delta t^{2}}m(\bv;\bu_{n+1}^{k})+\frac{\gamma\,(1-\alpha_{f})}{\beta\,\Delta t}c(\bv;\bu_{n+1}^{k})+\left(1-\alpha_{f}\right)k(\bv;\bu_{n+1}^{k})-\widetilde{\widetilde{l}}(\bv)\label{eq:nl_residual}
\end{equation}
and:
\begin{multline}
R'(\delta{\bu_{n+1}^{k+1}},\bv;\bu_{n+1}^{k})= \left[ \frac{1-\alpha_{m}}{\beta\,\Delta t^{2}}+ \frac{\gamma\,(1-\alpha_{f})\eta_M}{\beta\,\Delta t}  \right] M(\delta{\bu_{n+1}^{k+1}},\bv) \\ + (1-\alpha_{f})\left(1+\frac{\gamma\eta_{K}}{\beta\,\Delta t}\right)K_{t}(\delta{\bu_{n+1}^{k+1}},\bv;\bu_{n+1}^{k})\label{eq:linearization_1}
\end{multline}
with:
\begin{equation}
K_{t}(\delta{\bu_{n+1}^{k+1}},\bv;\bu_{n+1}^{k})=\int_{\Omega}\bepsilon(\delta{\bu_{n+1}^{k+1}}):\mathbb{D}^{k}:\bepsilon(\bv)\,{\rm d}V
\end{equation}
The fourth order stiffness tensor $\mathbb{D}^{k}=D_{ijkl}(\be_{i}\otimes\be_{j}\otimes\be_{k}\otimes\be_{l})$
can be defined differently according to the chosen algorithm. For
instance, if a standard Newton--Raphson formulation is chosen, $\mathbb{D}^{k}=\partial_{\bepsilon}\bsigma({\bu_{n+1}^{k}})$
is the tangent stiffness operator at iteration $k$.

Finally, the discretized variational formulation to solve becomes:
\begin{equation}
\boxed{\begin{aligned} & \text{Find}~{\delta{\bu_{n+1}^{k+1}}}~{\in\mathcal{U}_{\delta}}~~\text{such that : }\\
 & \left[ \frac{1-\alpha_{m}}{\beta\,\Delta t^{2}}+ \frac{\gamma\,(1-\alpha_{f})\eta_M}{\beta\,\Delta t}  \right] M(\delta{\bu_{n+1}^{k+1}},\bv) + (1-\alpha_{f})\left(\frac{\gamma}{\beta\,\Delta t}\eta_{K}+1\right)K_{d}(\delta{\bu_{n+1}^{k+1}},\bv;\bu_{n+1}^{k}) \\ & =-r(\bv;\bu_{n+1}^{k})\quad{{\forall\,}}\bv\in\mathcal{V}
\end{aligned}
}
\end{equation}
where $\mathcal{U}_{\delta}$ is the admissibility space of the displacement
variations, and $-r(\bv;\bu_{n+1}^{k})$ can now be interpreted as
the difference between the pseudo-external forces (represented by the linear form $\widetilde{\widetilde{l}}(\bv)$)
and the internal forces (first three terms of equation (\ref{eq:nl_residual})).

\section{Paraxial formulation for absorbing layers}

When spatially unbounded (infinite) domains are represented through bounded computational domains, spurious wave reflections can be observed boundaries.
Several techniques have been proposed in the literature to introduce proper treatments of the boundary conditions allowing to reproduce propagation processes in infinite one-phase and two-phase media artificially. Among the available formulations, one can cite the ones based upon using lumped dumpers \cite{lysmer1969finite}, Perfectly Matched Layers
(PML) \cite{berenger1994perfectly} and paraxial boundaries \cite{engquist1977absorbing,clayton1977absorbing,aubry1985local,MODARESSI1994117}.

\subsection{Standard formulation}

Paraxial approximation constitutes a local boundary condition which permits diffracting waves to be evacuated from the computational domain. To introduce the formulation, let us consider a split of the total domain $\Omega^{\infty}$ into two subdomains $\Omega$ and $\Omega^{E}$ separated by a surface $\Sigma\in\mathbb{R}^{d}$ of outer normal $\bmm$ (pointing from $\Omega$ to $\Omega^{E}$). On surface $\Sigma$, the continuity condition of the displacement field read:
\begin{equation}
[|\bu|]=\bu-\bu^{E}=0\qquad\Sigma\times[0,T]
\end{equation}
%
whereas the continuity of the traction vector reads:
%
\begin{equation}
[|\bsigma|]\cdot\bmm=\left(\bsigma-\bsigma^{E}\right)\cdot\bmm=\bsigma\cdot\bmm+\bsigma^{E}\cdot(-\bmm)=\bt+\bt^{E}=0\qquad\Sigma\times[0,T]\label{eq:paraxial_continuity_t}
\end{equation}

In previous equations, symbol $[|\bullet|]$ is used to denote the jump of function $\bullet$ across surface $\Sigma$, $\bu^{E}=\bu^{E}(\bx,t):\Omega^{E}\times[0,T]\to\mathbb{R}^{d}$ is the vector-valued displacement field on $\Omega^{E}$, and $\bsigma^{E}=\bsigma^{E}(\bu^{E}):\Omega^{E}\times[0,T]\to\mathbb{R}^{d\times d}$ is the corresponding stress tensor.

\subsubsection{Variational formulation}

Given the traction continuity condition (\ref{eq:paraxial_continuity_t}),
the variational problem to solve on $\Omega$ reads:
\begin{equation}
\boxed{\begin{aligned} & \text{Find}~{\bu}~{\in}~{\mathcal{U}}~~\text{such that: }\\
 & M(\ddot{\bu},\bv)+C(\dot{\bu},\bv)+K(\bu,\bv)=b(\bv;\bb)_{\Omega}+b(\bv;\bt)_{\partial_{t}\Omega}-b(\bv;\bt^{E})_{\Sigma}\quad\forall\,\bv\in\mathcal{V}
\end{aligned}
}\label{eq:paraxial_varf_1}
\end{equation}
Using a zeroth-order paraxial approximation, the traction vector $\bt^{E}$ can be written as:
\begin{equation}
\bt^{E}=A_{0}(\dot{\bu}) = \rho c_p \dot{u}_m \bmm + \rho c_s \dot{\bu}_s
\label{eq:paraxial_lin_form}
\end{equation}
where $(c_p,c_s)$ are the the propagation velocities of compressional and shear waves, $\dot{u}_m = \dot{\bu} \cdot \bmm$ is the velocity normal to $\Sigma$ and  $\dot{\bu}_s = \dot{\bu} - \dot{u}_m \bmm$ its tangent counterpart.

More in general, given a vector-valued field $\ba$ (e.g., the displacement, velocity and acceleration fields), we write:\footnote{We recall that given three Euclidean vectors $\bv \in \mathbb{R}^d$, $\bw\in \mathbb{R}^d$ and $\bz\in \mathbb{R}^d$, the dyadic product $\bv \otimes \bw \in \mathbb{R}^{d\times d}$ is the second order tensor defined by: $(\bv \otimes \bw)\cdot \bz = (\bw\cdot \bz) \bv $. In components: $(\bv \otimes \bw)_{ij}=v_i\,w_j$.}
\begin{equation}
    A_{0}(\ba) = \rho c_p \, (\bmm \otimes \bmm) \cdot \ba + \rho c_s ({\bI} - \bmm \otimes \bmm) \cdot \ba = \rho \left[ (c_p - c_s)\, m_i m_j + c_s \delta_{ij} \right] a_j \be_i
\end{equation}
where symbol $\otimes$ denotes the dyadic product between first order tensors (vectors).\footnote{Denoting $(m_x,m_y,m_z)$ the components of vector $\bmm$ in the reference system $R(O,\be_x,\be_y,\be_z)$, the components of $A_{0}(\ba)$ read:
\begin{align}
    (A_{0}(\ba))_x & = \rho \left[(c_p-c_s) m_x (m_x a_x + m_y a_y + m_z a_z) + c_s\,a_x \right] \\
    (A_{0}(\ba))_y & =  \rho \left[(c_p-c_s) m_y (m_x a_x + m_y a_y + m_z a_z) + c_s\,a_y \right] \\
    (A_{0}(\ba))_z & =  \rho \left[(c_p-c_s) m_z (m_x a_x + m_y a_y + m_z a_z) + c_s\,a_z \right]
\end{align}}

\subsubsection{Time-discretization}
After time discretization, equation \eqref{eq:paraxial_lin_form} becomes:
\begin{equation}
\begin{aligned}b(\bv;\bt^{E})_{\Sigma} & =\frac{\gamma\,(1-\alpha_{f})}{\beta\,\Delta t}b(\bv;A_{0}(\bu_{n+1}))_{\Sigma}-\left[\frac{\gamma(1-\alpha_{f})}{\beta}-1\right]b(\bv;A_{0}(\dot{\bu}_{n}))_{\Sigma}\\
 & \quad\cdots-\Delta t(1-\alpha_{f})\left(\frac{\gamma}{2\beta}-1\right)b(\bv;A_{0}(\ddot{\bu}_{n}))_{\Sigma}-\frac{\gamma(1-\alpha_{f})}{\beta\,\Delta t}b(\bv;A_{0}({\bu_{n}}))_{\Sigma}
\end{aligned}
\end{equation}
The variational problem to solve thus reads:
\begin{equation}
\boxed{\begin{aligned} & \text{Find} {\bu_{n+1}}~~\text{such that : }\\
 & \widetilde{\widetilde{\widetilde{K}}}({\bu_{n+1}},\bv)({\bu_{n+1}},\bv)=\widetilde{\widetilde{\widetilde{l}}}(\bv)\quad{{\forall\,}}\bv\in\mathcal{V}
\end{aligned}
}
\end{equation}
where $\widetilde{\widetilde{\widetilde{K}}}({\bu_{n+1}},\bv)$ is:
\begin{equation}
\begin{aligned}\widetilde{\widetilde{\widetilde{K}}}({\bu_{n+1}},\bv) & =\widetilde{\widetilde{K}}({\bu_{n+1}},\bv)+\frac{\gamma\,(1-\alpha_{f})}{\beta\,\Delta t}b(\bv;A_{0}(\bu_{n+1}))_{\Sigma}\\
 & =\frac{1-\alpha_{m}}{\beta\Delta t^{2}}M({\bu_{n+1}},\bv)+\frac{\gamma\,(1-\alpha_{f})}{\beta\,\Delta t}C({\bu_{n+1}},\bv)+\left(1-\alpha_{f}\right)K({\bu_{n+1}},\bv)\\
 & \quad\cdots+\frac{\gamma\,(1-\alpha_{f})}{\beta\,\Delta t}b(\bv;A_{0}(\bu_{n+1}))_{\Sigma}
\end{aligned}
\end{equation}
and $\widetilde{\widetilde{\widetilde{l}}}(\bv)$ is:
\begin{equation}
\begin{aligned}\widetilde{\widetilde{\widetilde{l}}}(\bv) & =\widetilde{\widetilde{l}}(\bv)+\left[\frac{\gamma(1-\alpha_{f})}{\beta}-1\right]b(\bv;A_{0}(\dot{\bu}_{n}))_{\Sigma}\\
 & \quad\cdots+\Delta t(1-\alpha_{f})\left(\frac{\gamma}{2\beta}-1\text{}\right)b(\bv;A_{0}(\ddot{\bu}_{n}))_{\Sigma}+\frac{\gamma(1-\alpha_{f})}{\beta\,\Delta t}b(\bv;A_{0}({\bu_{n}}))_{\Sigma}
\end{aligned}
\end{equation}

\subsection{Accounting for incident waves}
%
Let us now split the total displacement vector at $\Sigma$ into its incident $\bu_{in}$ and radiant $\bu_{r}$ components:
%
\begin{equation}
\bu=\bu^{E}=\bu_{in}+\bu_{r}
\end{equation}
%
and use the zeroth-order paraxial approximation for expressing the traction contribution to the traction vector $\bt^{E}$ due to the radiant field. The traction continuity condition (\ref{eq:paraxial_continuity_t}),
together with the linearity hypotheses at the vicinity of $\Sigma$, enables us to write:
%
\begin{equation}
\bt=-\bt^{E}(\bu^{E})=-\bt^{E}(\bu_{in})-\bt^{E}(\bu_{r})=-\bt^{E}(\bu_{in})-A_{0}(\dot{\bu}_{r})=-\bt^{E}(\dot{\bu}_{in})-A_{0}(\dot{\bu})+A_{0}(\dot{\bu}_{in})
\end{equation}
%
here $\dot{\bu}_{in}$ is known, whereas $\dot{\bu}$ is the unknown velocity field. 

The variational equation to solve now reads:
\begin{equation}
\boxed{\begin{aligned} & \text{Find}~{\bu}~{\in}~{\mathcal{U}}~~\text{such that : }\\
 & M(\ddot{\bu},\bv)+C(\dot{\bu},\bv)+K(\bu,\bv)=b(\bv;\bb)_{\Omega}+b(\bv;\bt)_{\partial_{t}\Omega}-b(\bv;A_{0}(\dot{\bu}))_{\Sigma}\\
 & \quad\cdots-b(\bv;\bt_{E}(\dot{\bu}_{in}))_{\Sigma}+b(\bv;A_{0}(\dot{\bu}_{in}))_{\Sigma}\quad\forall\,\bv\in\mathcal{V}
\end{aligned}
}
\end{equation}
%
where the last two terms are the only novelty with respect to equation \eqref{eq:paraxial_varf_1}.
