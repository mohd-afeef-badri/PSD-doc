PSD is a cross-platform solver built to work with Linux platforms
(Windows comping up soon). PSD has successfully been deployed on the
following platforms, CentOS 7/8, Ubuntu 16.04/18.04/20.04, Raspberry Pi,
Fedora 32, etc. Before installing PSD please ensure that you have the
following dependencies installed on your OS.

\textbf{Dependencies}

\begin{itemize}
\tightlist
\item
  \href{https://www.gnu.org/software/automake/}{automake}
\item
  \href{https://freefem.org/}{FreeFEM}
\item
  \href{https://www.mcs.anl.gov/petsc/}{PETSc}
\item
  \href{http://gmsh.info/}{Gmsh}
\item
  \href{http://www.cplusplus.com/}{C++}
\item
  \href{https://git-scm.com/}{git}
\item
  \href{http://tfel.sourceforge.net/}{MFront} (optional)
\item
  \href{https://thelfer.github.io/mgis/web/bindings-cxx.html}{MGIS}
  (optional)
\item
  \href{http://www.gnuplot.info/}{gnuplot} (optional)
\end{itemize}

\textbf{Now that I have all the dependencies what next}

\begin{itemize}
\item
  Go ahead and grab the latest copy of PSD. The code is hosted on GitLab
  \href{https://gitlab.com/PsdSolver/psd_sources}{repository}.

\begin{lstlisting}[language=bash]
git clone https://gitlab.com/PsdSolver/psd_sources.git PSD-Sources
\end{lstlisting}

  \textbf{Note}: You can also use SSH protocol if your key has been
  added to the repo in that case use

\begin{lstlisting}[language=bash]
git clone git@gitlab.com:PsdSolver/psd_sources.git
\end{lstlisting}
\item
  Use automake PSD within the cloned folder

\begin{lstlisting}[language=bash]
autoreconf -i
\end{lstlisting}
\item
  Configure PSD within the cloned folder

\begin{lstlisting}[language=bash]
./configure 
\end{lstlisting}

  \textbf{Note}: \lstinline!./configure! will install PSD in
  \lstinline!/usr/local/bin! and you would need sudo rights to perform
  installation, for non sudo users or for local install consider
  changing directory of installation. To change this directory use
  \lstinline!--prefix=Your/Own/Path! with \lstinline!./configure!.
  Remember to add \lstinline!Your/Own/Path! to your \lstinline!$PATH!
  variable, you can do so by
  \lstinline!export PATH=$PATH:Your/Own/Path!. Also
  \lstinline!./configure! will try to look for installation of
  \href{https://freefem.org/}{FreeFEM} and
  \href{http://gmsh.info/}{Gmsh} in \lstinline!usr/bin/! or
  \lstinline!usr/local/bin/! directories. If you have these packages
  installed in some other directory this should be specified during
  \lstinline!./configure! by using flags \lstinline!--with-FreeFEM=! and
  \lstinline!--with-Gmsh=!. For example, if
  \href{https://freefem.org/}{FreeFEM} is installed at
  \lstinline!home/FreeFem/bin! and \href{http://gmsh.info/}{Gmsh} in
  \lstinline!home/Gmsh/bin! then one should use

\begin{lstlisting}[language=bash]
./configure --with-FreeFEM=home/FreeFem/bin  --with-Gmsh=home/Gmsh/bin
\end{lstlisting}
\item
  Make PSD directives

\begin{lstlisting}[language=bash]
make
\end{lstlisting}
\item
  Install PSD

\begin{lstlisting}[language=bash]
sudo make install
\end{lstlisting}

  \textbf{Note} : You should not use \lstinline!sudo! if you have used
  \lstinline!--prefix! during the \lstinline!./configure!
\item
  Install PSD tutorials

\begin{lstlisting}[language=bash]
make tutorials
\end{lstlisting}
\end{itemize}

Now you should have the PSD solver installed on your machine. Note that,
the solver will be installed at \lstinline!usr/bin! or
\lstinline!usr/local/bin! directories if you used
\lstinline!sudo make install! or else it will be in your
\lstinline!--prefix! location. The PSD tutorials are installed in
\lstinline!$HOME/PSD-tutorials!.

\textbf{Additional FreeFEM tweak for brittle fracture mechanics}

Note that this procedure is only recommended if you are interested in
using PSD for brittle fracture problems. In your FreeFEM source files
(installation) go to \lstinline!src/femlib/fem.cpp! , in this file
replace the lines of code

\begin{lstlisting}[language=bash]
R seuil=hm/splitmax/4.0;
\end{lstlisting}

by the following

\begin{lstlisting}[language=bash]
R seuil=hm/splitmax/4.0/1000.0;
\end{lstlisting}

\subsubsection{Additional Installation steps for experts and
developers}\label{additional-installation-steps-for-experts-and-developers}

\begin{itemize}
\item
  Check if installation is correct

\begin{lstlisting}[language=bash]
make check
\end{lstlisting}
\item
  Configure PSD with MFront support

\begin{lstlisting}[language=bash]
./configure --prefix=$HOME/Install/local --with-mgis=$HOME/Install/local/mgis --with-mfront=$HOME/Install/local/mfront/bin/mfront  --with-FreeFEM=$HOME/Install/local/FreeFem/bin  --with-Gmsh=$HOME/Install/local/Gmsh/bin
\end{lstlisting}
\end{itemize}

Here in this example we assume that MFront, MGIS, Gmsh, and FreeFEM are
installed in \lstinline!$HOME/Install/local! directory, if that is not
the case for you please improvise or you can always ask for help from
Linux geeks around.

\begin{itemize}
\item
  Update PSD to the latest version. If you would like to update your old
  PSD source to a new one. Go to your\lstinline!PSD-Sources! folder and

\begin{lstlisting}[language=bash]
git pull origin master && ./reconfigure &&  make &&  make install
\end{lstlisting}
\item
  If you would like to install a developers copy of PSD

\begin{lstlisting}[language=bash]
make install-devel
\end{lstlisting}
\end{itemize}

\subsection{A quick sneak-peek of a typical PSD
simulation}\label{a-quick-sneak-peek-of-a-typical-psd-simulation}

PSD is a TUI (terminal user interface) based finite element solver.
Parallel or sequential PSD simulations can run on Linux platforms.
Command line options (flags) which user enters are used to control the
PSD solver. In order to make your choice of physics, model, mesh, etc.,
command line options need to be typed right into the bash.

A typical PSD simulation is performed in three steps.

\textbf{Step 1: Setting up the solver}

Its time to set up the PSD solver. Open the \lstinline!terminal! window
at the location of the solver, i.e., \lstinline!$HOME/PSD/Solver.! Then
run the following command in the \lstinline!terminal!.

\begin{lstlisting}[language=bash]
PSD_PreProcess [Options-PSD]
\end{lstlisting}

Via the command line options you will embed the physics within the
solver. This step generates a bunch of \lstinline!.edp! files which are
native to \href{https://freefem.org/}{FreeFEM} and additionally prints
out instructions on what to do next. You then need to open and edit
couple of these files via your favourite text editor, which could be
\lstinline!vim!, \lstinline!gedit! ,\lstinline!Notepad++!, etc. To
facilitate the edit process for your will have to go through the
instructions printed on the terminal.

For example to generate a sequential 2D elasticity solver for a problem
with body force and one Dirichlet border use

\begin{lstlisting}[language=bash]
PSD_PreProcess -dimension 2 -bodyforceconditions 1 -dirichletconditions 1
\end{lstlisting}

\textbf{Step 2: Launching the solver}

Now you are all set to run your simulation. To do so you will need to do
the run the following in the \lstinline!terminal!:

if you complied a parallel PSD version

\begin{lstlisting}[language=bash]
PSD_Solve -np $N Main.edp -v 0 -nw
\end{lstlisting}

if you complied a sequential PSD version

\begin{lstlisting}[language=bash]
PSD_Solve_Seq Main.edp -v 0 -nw
\end{lstlisting}

\begin{itemize}
\item
  In the parallel command \textbf{\$N} is an \lstinline!int! value,
  i.e., number of processes that you want to use for performing the
  simulation in parallel.
\item
  Additional flag \lstinline!-wg! may be required while launching the
  solver, this is in case debug mode is on.
\end{itemize}

\textbf{Step 3: Result visualization} Final step is to have a look at
the results of the simulation. PSD can provides output results in the
form of plots, finite element fields of interest, etc. ParaView's pvd,
vtu, and pvtu files are used for postprocessing. ASCII data files that
to trace certain quantities of interest like reaction forces, kinetic
energies, etc can also be outputted.

\subsection{PSD flags explained}\label{psd-flags-explained}

These are a set of commandline flags/options that control your
simulation. You can think of it as a way to talk to the solver. Here is
a table that lists out some of the options that are available (for full
list see documentation). It is advised to print these and have them
around when performing a PSD simulation.

\begin{longtable}[]{@{}lll@{}}
\toprule
\begin{minipage}[b]{0.26\columnwidth}\raggedright\strut
Flag\strut
\end{minipage} & \begin{minipage}[b]{0.09\columnwidth}\raggedright\strut
Type\strut
\end{minipage} & \begin{minipage}[b]{0.56\columnwidth}\raggedright\strut
Comment\strut
\end{minipage}\tabularnewline
\midrule
\endhead
\begin{minipage}[t]{0.26\columnwidth}\raggedright\strut
\textbf{Boolean flags}\strut
\end{minipage} & \begin{minipage}[t]{0.09\columnwidth}\raggedright\strut
\strut
\end{minipage} & \begin{minipage}[t]{0.56\columnwidth}\raggedright\strut
These flags accept values
\emph{1\textbar{}0\textbar{}yes\textbar{}no\textbar{}on\textbar{}off\textbar{}true\textbar{}false}
and are used to activate or deactivate any functionality of PSD.\strut
\end{minipage}\tabularnewline
\begin{minipage}[t]{0.26\columnwidth}\raggedright\strut
\strut
\end{minipage} & \begin{minipage}[t]{0.09\columnwidth}\raggedright\strut
\strut
\end{minipage} & \begin{minipage}[t]{0.56\columnwidth}\raggedright\strut
\strut
\end{minipage}\tabularnewline
\begin{minipage}[t]{0.26\columnwidth}\raggedright\strut
\lstinline!-help!\strut
\end{minipage} & \begin{minipage}[t]{0.09\columnwidth}\raggedright\strut
\lstinline![bool]!\strut
\end{minipage} & \begin{minipage}[t]{0.56\columnwidth}\raggedright\strut
To activate helping message on the terminal. Gives description and list
of available flags.\strut
\end{minipage}\tabularnewline
\begin{minipage}[t]{0.26\columnwidth}\raggedright\strut
\lstinline!-debug!\strut
\end{minipage} & \begin{minipage}[t]{0.09\columnwidth}\raggedright\strut
\lstinline![bool]!\strut
\end{minipage} & \begin{minipage}[t]{0.56\columnwidth}\raggedright\strut
To activate live plot while PSD runs. Development flag.\strut
\end{minipage}\tabularnewline
\begin{minipage}[t]{0.26\columnwidth}\raggedright\strut
\lstinline!-useGFP!\strut
\end{minipage} & \begin{minipage}[t]{0.09\columnwidth}\raggedright\strut
\lstinline![bool]!\strut
\end{minipage} & \begin{minipage}[t]{0.56\columnwidth}\raggedright\strut
To activate use of GoFastPlugins. A suite of C++ plugins.\strut
\end{minipage}\tabularnewline
\begin{minipage}[t]{0.26\columnwidth}\raggedright\strut
\lstinline!-useRCM!\strut
\end{minipage} & \begin{minipage}[t]{0.09\columnwidth}\raggedright\strut
\lstinline![bool]!\strut
\end{minipage} & \begin{minipage}[t]{0.56\columnwidth}\raggedright\strut
Activate mesh level renumbering: Reverse Cuthill Mckee.\strut
\end{minipage}\tabularnewline
\begin{minipage}[t]{0.26\columnwidth}\raggedright\strut
\lstinline!-pipegnu!\strut
\end{minipage} & \begin{minipage}[t]{0.09\columnwidth}\raggedright\strut
\lstinline![bool]!\strut
\end{minipage} & \begin{minipage}[t]{0.56\columnwidth}\raggedright\strut
Use to activate real time pipe plotting using
\href{http://www.gnuplot.info/}{gnuplot}.\strut
\end{minipage}\tabularnewline
\begin{minipage}[t]{0.26\columnwidth}\raggedright\strut
\lstinline!-timelog!\strut
\end{minipage} & \begin{minipage}[t]{0.09\columnwidth}\raggedright\strut
\lstinline![bool]!\strut
\end{minipage} & \begin{minipage}[t]{0.56\columnwidth}\raggedright\strut
To activate time logging the different phases of the solver.\strut
\end{minipage}\tabularnewline
\begin{minipage}[t]{0.26\columnwidth}\raggedright\strut
\lstinline!-supercomp!\strut
\end{minipage} & \begin{minipage}[t]{0.09\columnwidth}\raggedright\strut
\lstinline![bool]!\strut
\end{minipage} & \begin{minipage}[t]{0.56\columnwidth}\raggedright\strut
Use when using a super computer without Xterm support\strut
\end{minipage}\tabularnewline
\begin{minipage}[t]{0.26\columnwidth}\raggedright\strut
\lstinline!-bodyforce!\strut
\end{minipage} & \begin{minipage}[t]{0.09\columnwidth}\raggedright\strut
\lstinline![bool]!\strut
\end{minipage} & \begin{minipage}[t]{0.56\columnwidth}\raggedright\strut
To activate volumetric source term (body force).\strut
\end{minipage}\tabularnewline
\begin{minipage}[t]{0.26\columnwidth}\raggedright\strut
\lstinline!-vectorial!\strut
\end{minipage} & \begin{minipage}[t]{0.09\columnwidth}\raggedright\strut
\lstinline![bool]!\strut
\end{minipage} & \begin{minipage}[t]{0.56\columnwidth}\raggedright\strut
To use vectorial finite element method.\strut
\end{minipage}\tabularnewline
\begin{minipage}[t]{0.26\columnwidth}\raggedright\strut
\lstinline!-pointprobe!\strut
\end{minipage} & \begin{minipage}[t]{0.09\columnwidth}\raggedright\strut
\lstinline![bool]!\strut
\end{minipage} & \begin{minipage}[t]{0.56\columnwidth}\raggedright\strut
To postprocess point fields.\strut
\end{minipage}\tabularnewline
\begin{minipage}[t]{0.26\columnwidth}\raggedright\strut
\lstinline!-sequential!\strut
\end{minipage} & \begin{minipage}[t]{0.09\columnwidth}\raggedright\strut
\lstinline![bool]!\strut
\end{minipage} & \begin{minipage}[t]{0.56\columnwidth}\raggedright\strut
To solve via a sequential solver.\strut
\end{minipage}\tabularnewline
\begin{minipage}[t]{0.26\columnwidth}\raggedright\strut
\lstinline!-energydecomp!\strut
\end{minipage} & \begin{minipage}[t]{0.09\columnwidth}\raggedright\strut
\lstinline![bool]!\strut
\end{minipage} & \begin{minipage}[t]{0.56\columnwidth}\raggedright\strut
To activate energy decomposition, only for phase-field.\strut
\end{minipage}\tabularnewline
\begin{minipage}[t]{0.26\columnwidth}\raggedright\strut
\lstinline!-doublecouple!\strut
\end{minipage} & \begin{minipage}[t]{0.09\columnwidth}\raggedright\strut
\lstinline![bool]!\strut
\end{minipage} & \begin{minipage}[t]{0.56\columnwidth}\raggedright\strut
To activate double couple source for soildynamics.\strut
\end{minipage}\tabularnewline
\begin{minipage}[t]{0.26\columnwidth}\raggedright\strut
\lstinline!-constrainHPF!\strut
\end{minipage} & \begin{minipage}[t]{0.09\columnwidth}\raggedright\strut
\lstinline![bool]!\strut
\end{minipage} & \begin{minipage}[t]{0.56\columnwidth}\raggedright\strut
To use constrain condition in hybrid phase-field model.\strut
\end{minipage}\tabularnewline
\begin{minipage}[t]{0.26\columnwidth}\raggedright\strut
\lstinline!-top2vol-meshing!\strut
\end{minipage} & \begin{minipage}[t]{0.09\columnwidth}\raggedright\strut
\lstinline![bool]!\strut
\end{minipage} & \begin{minipage}[t]{0.56\columnwidth}\raggedright\strut
Activate top-ii-vol point source meshing for soil-dynamics.\strut
\end{minipage}\tabularnewline
\begin{minipage}[t]{0.26\columnwidth}\raggedright\strut
\lstinline!-getreactionforce!\strut
\end{minipage} & \begin{minipage}[t]{0.09\columnwidth}\raggedright\strut
\lstinline![bool]!\strut
\end{minipage} & \begin{minipage}[t]{0.56\columnwidth}\raggedright\strut
Activate routine for extraction reactions at surface.\strut
\end{minipage}\tabularnewline
\begin{minipage}[t]{0.26\columnwidth}\raggedright\strut
\lstinline!-plotreactionforce!\strut
\end{minipage} & \begin{minipage}[t]{0.09\columnwidth}\raggedright\strut
\lstinline![bool]!\strut
\end{minipage} & \begin{minipage}[t]{0.56\columnwidth}\raggedright\strut
Activate realtime pipe plotting using GnuPlot.\strut
\end{minipage}\tabularnewline
\begin{minipage}[t]{0.26\columnwidth}\raggedright\strut
\lstinline!-withmaterialtensor!\strut
\end{minipage} & \begin{minipage}[t]{0.09\columnwidth}\raggedright\strut
\lstinline![bool]!\strut
\end{minipage} & \begin{minipage}[t]{0.56\columnwidth}\raggedright\strut
Activate material tensor for building stiffness matrix.\strut
\end{minipage}\tabularnewline
\begin{minipage}[t]{0.26\columnwidth}\raggedright\strut
\lstinline!-crackdirichletcondition!\strut
\end{minipage} & \begin{minipage}[t]{0.09\columnwidth}\raggedright\strut
\lstinline![bool]!\strut
\end{minipage} & \begin{minipage}[t]{0.56\columnwidth}\raggedright\strut
To activate pre-cracked surface Dirichlet.\strut
\end{minipage}\tabularnewline
\begin{minipage}[t]{0.26\columnwidth}\raggedright\strut
\strut
\end{minipage} & \begin{minipage}[t]{0.09\columnwidth}\raggedright\strut
\strut
\end{minipage} & \begin{minipage}[t]{0.56\columnwidth}\raggedright\strut
\strut
\end{minipage}\tabularnewline
\begin{minipage}[t]{0.26\columnwidth}\raggedright\strut
\textbf{Integer flags}\strut
\end{minipage} & \begin{minipage}[t]{0.09\columnwidth}\raggedright\strut
\strut
\end{minipage} & \begin{minipage}[t]{0.56\columnwidth}\raggedright\strut
These flags accept a integer value followed by the flag itself. These
integer values are used in PSD simulations for various
definitions.\strut
\end{minipage}\tabularnewline
\begin{minipage}[t]{0.26\columnwidth}\raggedright\strut
\strut
\end{minipage} & \begin{minipage}[t]{0.09\columnwidth}\raggedright\strut
\strut
\end{minipage} & \begin{minipage}[t]{0.56\columnwidth}\raggedright\strut
\strut
\end{minipage}\tabularnewline
\begin{minipage}[t]{0.26\columnwidth}\raggedright\strut
\lstinline!-dirichletpointconditions!\strut
\end{minipage} & \begin{minipage}[t]{0.09\columnwidth}\raggedright\strut
\lstinline![int]!\strut
\end{minipage} & \begin{minipage}[t]{0.56\columnwidth}\raggedright\strut
Number of Dirichlet points. \strut
\end{minipage}\tabularnewline
\begin{minipage}[t]{0.26\columnwidth}\raggedright\strut
\lstinline!-dirichletconditions!\strut
\end{minipage} & \begin{minipage}[t]{0.09\columnwidth}\raggedright\strut
\lstinline![int]!\strut
\end{minipage} & \begin{minipage}[t]{0.56\columnwidth}\raggedright\strut
Number of Dirichlet boundaries.\strut
\end{minipage}\tabularnewline
\begin{minipage}[t]{0.26\columnwidth}\raggedright\strut
\lstinline!-bodyforceconditions!\strut
\end{minipage} & \begin{minipage}[t]{0.09\columnwidth}\raggedright\strut
\lstinline![int]!\strut
\end{minipage} & \begin{minipage}[t]{0.56\columnwidth}\raggedright\strut
Number of regions acted upon by bodyforce.\strut
\end{minipage}\tabularnewline
\begin{minipage}[t]{0.26\columnwidth}\raggedright\strut
\lstinline!-tractionconditions!\strut
\end{minipage} & \begin{minipage}[t]{0.09\columnwidth}\raggedright\strut
\lstinline![int]!\strut
\end{minipage} & \begin{minipage}[t]{0.56\columnwidth}\raggedright\strut
Number of Neumann/traction boundaries.\strut
\end{minipage}\tabularnewline
\begin{minipage}[t]{0.26\columnwidth}\raggedright\strut
\lstinline!-parmetis_worker!\strut
\end{minipage} & \begin{minipage}[t]{0.09\columnwidth}\raggedright\strut
\lstinline![int]!\strut
\end{minipage} & \begin{minipage}[t]{0.56\columnwidth}\raggedright\strut
Number of parallel workers used by ParMetis for partitioning.\strut
\end{minipage}\tabularnewline
\begin{minipage}[t]{0.26\columnwidth}\raggedright\strut
\lstinline!-lagrange!\strut
\end{minipage} & \begin{minipage}[t]{0.09\columnwidth}\raggedright\strut
\lstinline![int]!\strut
\end{minipage} & \begin{minipage}[t]{0.56\columnwidth}\raggedright\strut
Lagrange order used for FE spaces. 1 for P1 or 2 for P2.\strut
\end{minipage}\tabularnewline
\begin{minipage}[t]{0.26\columnwidth}\raggedright\strut
\lstinline!-dimension!\strut
\end{minipage} & \begin{minipage}[t]{0.09\columnwidth}\raggedright\strut
\lstinline![int]!\strut
\end{minipage} & \begin{minipage}[t]{0.56\columnwidth}\raggedright\strut
Accepts values 2 or 3. Use 3 for 3D. and 2 for 2D problem.\strut
\end{minipage}\tabularnewline
\begin{minipage}[t]{0.26\columnwidth}\raggedright\strut
\strut
\end{minipage} & \begin{minipage}[t]{0.09\columnwidth}\raggedright\strut
\strut
\end{minipage} & \begin{minipage}[t]{0.56\columnwidth}\raggedright\strut
\strut
\end{minipage}\tabularnewline
\begin{minipage}[t]{0.26\columnwidth}\raggedright\strut
\textbf{String flags}\strut
\end{minipage} & \begin{minipage}[t]{0.09\columnwidth}\raggedright\strut
\strut
\end{minipage} & \begin{minipage}[t]{0.56\columnwidth}\raggedright\strut
These flags accept a string value followed by the flag itself. These
string values are used in PSD simulations for various definitions.\strut
\end{minipage}\tabularnewline
\begin{minipage}[t]{0.26\columnwidth}\raggedright\strut
\strut
\end{minipage} & \begin{minipage}[t]{0.09\columnwidth}\raggedright\strut
\strut
\end{minipage} & \begin{minipage}[t]{0.56\columnwidth}\raggedright\strut
\strut
\end{minipage}\tabularnewline
\begin{minipage}[t]{0.26\columnwidth}\raggedright\strut
\lstinline!-mesh!\strut
\end{minipage} & \begin{minipage}[t]{0.09\columnwidth}\raggedright\strut
\lstinline![sting]!\strut
\end{minipage} & \begin{minipage}[t]{0.56\columnwidth}\raggedright\strut
Provide mesh to be solved by \lstinline!PSD_Solve!.\strut
\end{minipage}\tabularnewline
\begin{minipage}[t]{0.26\columnwidth}\raggedright\strut
\lstinline!-timediscretization!\strut
\end{minipage} & \begin{minipage}[t]{0.09\columnwidth}\raggedright\strut
\lstinline![sting]!\strut
\end{minipage} & \begin{minipage}[t]{0.56\columnwidth}\raggedright\strut
Time discretization type. Use ``generalized\_alpha'' or
``newmark\_beta'' or ``hht\_alpha'' or ``central\_difference''\strut
\end{minipage}\tabularnewline
\begin{minipage}[t]{0.26\columnwidth}\raggedright\strut
\lstinline!-nonlinearmethod!\strut
\end{minipage} & \begin{minipage}[t]{0.09\columnwidth}\raggedright\strut
\lstinline![sting]!\strut
\end{minipage} & \begin{minipage}[t]{0.56\columnwidth}\raggedright\strut
Nonlinear method type. Use ``Picard'' or ``Newton\_Raphsons''.\strut
\end{minipage}\tabularnewline
\begin{minipage}[t]{0.26\columnwidth}\raggedright\strut
\lstinline!-reactionforce!\strut
\end{minipage} & \begin{minipage}[t]{0.09\columnwidth}\raggedright\strut
\lstinline![sting]!\strut
\end{minipage} & \begin{minipage}[t]{0.56\columnwidth}\raggedright\strut
Reaction force calculation method ``stress\_based'' or
``variational\_based''.\strut
\end{minipage}\tabularnewline
\begin{minipage}[t]{0.26\columnwidth}\raggedright\strut
\lstinline!-doublecouple!\strut
\end{minipage} & \begin{minipage}[t]{0.09\columnwidth}\raggedright\strut
\lstinline![sting]!\strut
\end{minipage} & \begin{minipage}[t]{0.56\columnwidth}\raggedright\strut
Soil dynamics double couple. Use ``force\_based'' or
``displacement\_based''.\strut
\end{minipage}\tabularnewline
\begin{minipage}[t]{0.26\columnwidth}\raggedright\strut
\lstinline!-postprocess!\strut
\end{minipage} & \begin{minipage}[t]{0.09\columnwidth}\raggedright\strut
\lstinline![sting]!\strut
\end{minipage} & \begin{minipage}[t]{0.56\columnwidth}\raggedright\strut
To communicate what to postprocess ``u'', ``v'', ``a'', ``uv'' , ``ud'',
``ua'', ``d'', ``ud'', or ``uav''.\strut
\end{minipage}\tabularnewline
\begin{minipage}[t]{0.26\columnwidth}\raggedright\strut
\lstinline!-partitioner!\strut
\end{minipage} & \begin{minipage}[t]{0.09\columnwidth}\raggedright\strut
\lstinline![sting]!\strut
\end{minipage} & \begin{minipage}[t]{0.56\columnwidth}\raggedright\strut
Mesh partitioner could be
``\href{http://glaros.dtc.umn.edu/gkhome/metis/metis/overview}{metis}''
``\href{http://glaros.dtc.umn.edu/gkhome/metis/parmetis/overview}{parmetis}''
or ``\href{http://www.labri.fr/perso/pelegrin/scotch/}{scotch}''.\strut
\end{minipage}\tabularnewline
\begin{minipage}[t]{0.26\columnwidth}\raggedright\strut
\lstinline!-problem!\strut
\end{minipage} & \begin{minipage}[t]{0.09\columnwidth}\raggedright\strut
\lstinline![sting]!\strut
\end{minipage} & \begin{minipage}[t]{0.56\columnwidth}\raggedright\strut
Interested problem. Use ``linear\_elasticity'', ``damage'',
``elastodynamics'', or ``soildynamics''.\strut
\end{minipage}\tabularnewline
\begin{minipage}[t]{0.26\columnwidth}\raggedright\strut
\lstinline!-model!\strut
\end{minipage} & \begin{minipage}[t]{0.09\columnwidth}\raggedright\strut
\lstinline![sting]!\strut
\end{minipage} & \begin{minipage}[t]{0.56\columnwidth}\raggedright\strut
Interested model. Use ``hybrid\_phase\_field'' or ``Mazar''.\strut
\end{minipage}\tabularnewline
\bottomrule
\end{longtable}

\emph{To report bugs, issues, feature-requests contact:}

\begin{itemize}
\tightlist
\item
  \textbf{mohd-afeef.badri@cea.fr}
\item
  \textbf{mohd-afeef.badri@hotmail.com}
\end{itemize}
