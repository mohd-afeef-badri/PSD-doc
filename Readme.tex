\subsection{PSD version 2.4}

\begin{center}\rule{0.5\linewidth}{\linethickness}\end{center}

PSD, acronym for Parallel Solid/Structural/Seismic Dynamics, is a finite
elements-based solid mechanics solver with capabilities of performing
High Performance Computing (HPC) simulations with billions of unknowns.
The kernel of PSD is wrapped around \href{https://freefem.org/}{FreeFEM}
for finite element discretization, and
\href{https://www.mcs.anl.gov/petsc/}{PETSc} for linear
algebra/Preconditioning. PSD solver contains straightforward supports
for \emph{static} or \emph{dynamic} simulations with \emph{linear} and
\emph{nonlinear} solid mechanics problems. Besides these
\href{https://link.springer.com/article/10.1007/s00466-014-1109-y}{\emph{hybrid-phase
field fracture mechanics}} models have also been incorporated within
PSD. For dynamics the
\href{https://hal.archives-ouvertes.fr/hal-00345290/document}{\emph{genralized-\(\alpha\)
model}} for time discretization is used, this models enable
straightforward use of Newmark-\(\beta\), central difference, or HHT as
time discretization. PSD uses sate-of-the art domain-decomposition
paradigm via
\href{https://www.sciencedirect.com/science/article/pii/S0022407317309597}{\emph{vectorial
finite elements}} for parallel computing and all solvers are proven to
scale quasi-optimally. PSD has proven scalabilty uptill 24,000 cores
with largest problem solved containing over 5 Billion unknowns.

Besides the parallel suite, PSD also includes a sequential solver which
does not require \href{https://www.mcs.anl.gov/petsc/}{PETSc}.

PSD works for two and three dimensional problems only. Unstructured
meshes (triangular for 2D and tetrahedral for 3D) are supported in
\href{https://www.ljll.math.upmc.fr/frey/software.html}{MEDIT's}
\lstinline!.mesh! format or \href{http://gmsh.info/}{Gmsh's}
\lstinline!.msh! format. PSD post processing is done via
\lstinline!.pvd! ,\lstinline!.vtk! and \lstinline!.vtu! files of the
\href{https://www.paraview.org/}{ParaView} platform.

\subsection{Installation}

PSD is a cross-platform FEM solver built to work with Linux/Unix
platforms. PSD has successfully been deployed on the following
platforms:

\begin{itemize}
\tightlist
\item
  CentOS 7 / 8
\item
  Ubuntu 16.04 / 18.04 / 20.04 / 22.04
\item
  Raspberry Pi
\item
  Fedora 30 / 32 / 34
\item
  MacOS X Puma:10.1.5
\end{itemize}

Before installing PSD please ensure that you have the following
dependencies installed on your OS.

\textbf{Dependencies}

PSD has some essential prerequisites without which PSD will not
function. These dependencies can either be preinstalled by user. In that
case following list needs to be installed, and assured that these are
intercompatiable.

\begin{longtable}[]{@{}lll@{}}
\toprule
Package & Version & Essential\tabularnewline
\midrule
\endhead
\href{https://www.gnu.org/software/automake/}{automake} & Version
\textbf{2.8} or higher & YES\tabularnewline
\href{https://freefem.org/}{FreeFEM} & Version \textbf{4.10} &
YES\tabularnewline
\href{https://www.mcs.anl.gov/petsc/}{PETSc} & Version \textbf{3.16.1} &
YES\tabularnewline
\href{http://gmsh.info/}{Gmsh} & Version \textbf{4.8.4} &
YES\tabularnewline
\href{http://www.cplusplus.com/}{C++}, C & Version \textbf{7} or Higher
& YES\tabularnewline
MPI & Version \textbf{2.0} or Higher (choose either
\href{https://www.mpich.org/}{Mpich} or
\href{https://www.open-mpi.org/}{Open MPI}) & YES\tabularnewline
\href{https://git-scm.com/}{git} & - & YES\tabularnewline
\href{https://www.salome-platform.org/}{salome} & Version \textbf{9} or
Higher & OPTIONAL\tabularnewline
\href{http://tfel.sourceforge.net/}{MFront} & Version \textbf{3.4.0} &
OPTIONAL\tabularnewline
\href{https://thelfer.github.io/mgis/web/bindings-cxx.html}{MGIS} &
Version \textbf{1.2} & OPTIONAL\tabularnewline
\href{http://www.gnuplot.info/}{gnuplot} & Version \textbf{4.0} or
Higher & OPTIONAL\tabularnewline
& &\tabularnewline
\bottomrule
\end{longtable}

Note that PETSc and FreeFEM need to be compiled with METIS, ParMETIS,
and hpddm support, in this case these form extra dependencies. The user
in incharge of making sure that all the dependencies are met before
installing PSD. Have a look at \lstinline!Download.MD! file which
explains a little bit about the dependencies and also provides the
version and the links to download the dependencies from.

Generally there are two procedures of installing PSD.

\begin{itemize}
\tightlist
\item
  \emph{Installation Procedure 1: I install my own dependencies}.
\item
  \emph{Installation Procedure 2: PSD installs all dependencies}.
\end{itemize}

Among the two the second one is the recommended. With \emph{Installation
Procedure 1} user takes charge and installs all the dependencies. To
follow this type of installation read the section \emph{Installation
Procedure 1: I install my own dependencies}. Alternatively, PSD can
attempt to build and compile FreeFEM, PETSc, Gmsh, MFront, and MGIS for
you. In this case, user needs to ensure that automake, C/C++, git, and
MPI is available in the system and that your system has an active
internet connection. To know how this type of installation is done,
please skip to section \emph{Installation Procedure 2: PSD installs all
dependencies}.

\paragraph{Installation Procedure 1: I install my own dependencies}

\begin{itemize}
\item
  Grab the latest copy of PSD. The code is hosted on GitLab
  \href{https://gitlab.com/PsdSolver/psd_sources}{repository}.

\begin{lstlisting}[language=bash]
git clone https://gitlab.com/PsdSolver/psd_sources.git PSD-Sources
\end{lstlisting}
\item
  Use automake within the cloned PSD folder (\lstinline!psd_sources!)

\begin{lstlisting}[language=bash]
autoreconf -i
\end{lstlisting}
\item
  Configure PSD within the cloned folder

\begin{lstlisting}[language=bash]
./configure
\end{lstlisting}

  \textbf{Note}: \lstinline!./configure! will install PSD in
  \lstinline!/usr/local/bin! and you would need \lstinline!sudo! rights
  (superuser) to perform installation, for non \lstinline!sudo! users or
  for local install consider changing directory of installation. To
  change this directory use \lstinline!--prefix=Your/Own/Path! with
  \lstinline!./configure!. Remember to add \lstinline!Your/Own/Path! to
  your \lstinline!$PATH! variable, you can do so by
  \lstinline!export PATH=$PATH:Your/Own/Path!.
\end{itemize}

\textbf{Note}: \lstinline!./configure! will try to look for installation
of FreeFEM, MGIS, Mfront, and Gmsh in your \lstinline!PATH! directories.
If you have these packages installed in some other directory this should
be specified during \lstinline!./configure! by using flags
\lstinline!--with-FreeFEM=! , \lstinline!--with-Gmsh=!, etc. as shown
below. For example

\begin{lstlisting}[language=bash]
./configure --prefix=$HOME/Install/local \
  --with-mgis=$HOME/Install/local/mgis   \
  --with-mfront=$HOME/Install/local/mfront/bin/mfront  \
  --with-FreeFEM=$HOME/Install/local/FreeFem/bin       \
  --with-Gmsh=$HOME/Install/local/Gmsh/bin        
\end{lstlisting}

Additionally \lstinline!--with-salome=/salome/install/dir! can be used
for compiling PSD with SALOME support.

\begin{itemize}
\tightlist
\item
  Make PSD directives
\end{itemize}

\begin{lstlisting}[language=bash]
make
\end{lstlisting}

\begin{itemize}
\item
  Install PSD

\begin{lstlisting}[language=bash]
sudo make install
\end{lstlisting}

  \textbf{Note} : You should not use \lstinline!sudo! if you have used
  \lstinline!--prefix! during the \lstinline!./configure! phase.
\item
  Install PSD tutorials

\begin{lstlisting}[language=bash]
make tutorials
\end{lstlisting}
\end{itemize}

Now you should have the PSD solver installed on your machine. Note that,
the solver will be installed at \lstinline!usr/bin! or
\lstinline!usr/local/bin! directories if you used
\lstinline!sudo make install! or else it will be in your
\lstinline!--prefix! location. The PSD tutorials are installed in
\lstinline!$HOME/PSD-tutorials!.

\textbf{Additional FreeFEM tweak for brittle fracture mechanics}

Note that this procedure is only recommended if you are interested in
using PSD for brittle fracture problems. In your FreeFEM source files
(installation) go to \lstinline!src/femlib/fem.cpp! , in this file
replace the lines of code

\begin{lstlisting}[language=bash]
R seuil=hm/splitmax/4.0;
\end{lstlisting}

by the following

\begin{lstlisting}[language=bash]
R seuil=hm/splitmax/4.0/1000.0;
\end{lstlisting}

\paragraph{Installation Procedure 2: PSD installs all the dependencies}

\textbf{Note}: please make sure that you are using PSD version 2.5 or
above.

\begin{itemize}
\item
  Grab the latest copy of PSD. The code is hosted on GitLab
  \href{https://gitlab.com/PsdSolver/psd_sources}{repository}.

\begin{lstlisting}[language=bash]
git clone https://gitlab.com/PsdSolver/psd_sources.git PSD-Sources
\end{lstlisting}
\item
  We will install PSD and all its dependencies in folder
  \lstinline!/home/PSDinstall! . Let us start by making it a temporary
  environmental variable.

\begin{lstlisting}[language=bash]
export PREFIXPSD=/home/PSDinstall
\end{lstlisting}
\item
  Use \lstinline!automake! within the cloned folder and configure PSD

\begin{lstlisting}[language=bash]
autoreconf -i && ./configure --prefix=$PREFIXPSD --with-dependencies=yes
\end{lstlisting}

  \textbf{Note}: we will install PSD in \lstinline!/home/PSDinstall! and
  you would need read and write rights to perform installation in this
  folder. \lstinline!--with-dependencies=yes! uses \lstinline!wget! and
  internet, so make sure you are connected. To bypass the internet
  limitation, for instance on clusters or supercomputers, users can
  provide the tarball files \lstinline!*.tar.gz! files of the
  dependencies within the \lstinline!/ext! folder and use the flag
  \lstinline!--with-zipped_dependecies!.\\
  \textbf{Note}: for CentOS7 you might have to load MPI with
  \lstinline!module load mpi/openmpi-x86_64!
\item
  Make all dependencies

\begin{lstlisting}[language=bash]
cd ext && make && cd ..
\end{lstlisting}
\item
  The \lstinline!PATH! and \lstinline!LSD_LIBRARY_PATH! variables need
  to be updated. Add the following two line to your
  \lstinline!~/.bashrc!

\begin{lstlisting}[language=bash]
export PATH=/home/PSDInstall/bin:$PATH 
export LD_LIBRARY_PATH=/home/PSDInstall/lib:$LD_LIBRARY_PATH
\end{lstlisting}

  then do

\begin{lstlisting}[language=bash]
source ~/.bashrc
\end{lstlisting}

  \textbf{Note:} this can also be done temporarily by
  \lstinline!source $PREFIXPSD/mfront-env.sh!. If you follow this
  temporary approach, every time before using PSD you will need to redo
  this command.
\item
  Reconfigure PSD with the installed dependencies

\begin{lstlisting}[language=bash]
./configure --prefix=$PREFIXPSD \
  --with-mgis=$PREFIXPSD        \
  --with-mfront=$PREFIXPSD      \
  --with-FreeFEM=$PREFIXPSD/bin \
  --with-Gmsh=$PREFIXPSD/bin
\end{lstlisting}
\item
  Compile and install PSD

\begin{lstlisting}[language=bash]
make && make install
\end{lstlisting}
\item
  Perform a check to see if everything works

\begin{lstlisting}[language=bash]
make check
\end{lstlisting}
\item
  Install PSD tutorials

\begin{lstlisting}[language=bash]
make tutorials
\end{lstlisting}
\end{itemize}

Now you should have the PSD solver installed on your machine. Note that,
the solver will be installed at \lstinline!home/PSDInstall!. The PSD
tutorials are installed in \lstinline!$HOME/PSD-tutorials!.

\subsection{A quick sneak-peek of a typical PSD simulation}

PSD is a TUI (terminal user interface) based finite element solver.
Parallel or sequential PSD simulations can run on Linux platforms.
Command line options (flags) which user enters are used to control the
PSD solver. In order to make your choice of physics, model, mesh, etc.,
command line options need to be typed right into the bash.

A typical PSD simulation is performed in three steps.

\textbf{Step 1: Setting up the solver}

Its time to set up the PSD solver. Open the \lstinline!terminal! window
at the location of the solver, i.e., \lstinline!$HOME/PSD/Solver.! Then
run the following command in the \lstinline!terminal!.

\begin{lstlisting}[language=bash]
PSD_PreProcess [Options-PSD]
\end{lstlisting}

Via the command line options you will embed the physics within the
solver. This step generates a bunch of \lstinline!.edp! files which are
native to \href{https://freefem.org/}{FreeFEM} and additionally prints
out instructions on what to do next. You then need to open and edit
couple of these files via your favourite text editor, which could be
\lstinline!vim!, \lstinline!gedit! ,\lstinline!Notepad++!, etc. To
facilitate the edit process for your will have to go through the
instructions printed on the terminal.

For example to generate a sequential 2D elasticity solver for a problem
with body force and one Dirichlet border use

\begin{lstlisting}[language=bash]
PSD_PreProcess -dimension 2 -bodyforceconditions 1 -dirichletconditions 1
\end{lstlisting}

\textbf{Step 2: Launching the solver}

Now you are all set to run your simulation. To do so you will need to do
the run the following in the \lstinline!terminal!:

if you complied a parallel PSD version

\begin{lstlisting}[language=bash]
PSD_Solve -np $N Main.edp -v 0 -nw
\end{lstlisting}

if you complied a sequential PSD version

\begin{lstlisting}[language=bash]
PSD_Solve_Seq Main.edp -v 0 -nw
\end{lstlisting}

\begin{itemize}
\item
  In the parallel command \textbf{\$N} is an \lstinline!int! value,
  i.e., number of processes that you want to use for performing the
  simulation in parallel.
\item
  Additional flag \lstinline!-wg! may be required while launching the
  solver, this is in case debug mode is on.
\end{itemize}

\textbf{Step 3: Result visualization} Final step is to have a look at
the results of the simulation. PSD can provides output results in the
form of plots, finite element fields of interest, etc. ParaView's pvd,
vtu, and pvtu files are used for postprocessing (see figure below).
ASCII data files that to trace certain quantities of interest like
reaction forces, kinetic energies, etc can also be outputted.

\subsection{PSD flags explained}

These are a set of commandline flags/options that control your
simulation. You can think of it as a way to talk to the solver. Here is
a table that lists out some of the options that are available (for full
list see documentation). It is advised to print these and have them
around when performing a PSD simulation.

\begin{longtable}[]{@{}lll@{}}
\toprule
\begin{minipage}[b]{0.26\columnwidth}\raggedright\strut
Flag\strut
\end{minipage} & \begin{minipage}[b]{0.09\columnwidth}\raggedright\strut
Type\strut
\end{minipage} & \begin{minipage}[b]{0.56\columnwidth}\raggedright\strut
Comment\strut
\end{minipage}\tabularnewline
\midrule
\endhead
\begin{minipage}[t]{0.26\columnwidth}\raggedright\strut
\textbf{Boolean flags}\strut
\end{minipage} & \begin{minipage}[t]{0.09\columnwidth}\raggedright\strut
\strut
\end{minipage} & \begin{minipage}[t]{0.56\columnwidth}\raggedright\strut
These flags accept values
\emph{1\textbar{}0\textbar{}yes\textbar{}no\textbar{}on\textbar{}off\textbar{}true\textbar{}false}
and are used to activate or deactivate any functionality of PSD.\strut
\end{minipage}\tabularnewline
\begin{minipage}[t]{0.26\columnwidth}\raggedright\strut
\strut
\end{minipage} & \begin{minipage}[t]{0.09\columnwidth}\raggedright\strut
\strut
\end{minipage} & \begin{minipage}[t]{0.56\columnwidth}\raggedright\strut
\strut
\end{minipage}\tabularnewline
\begin{minipage}[t]{0.26\columnwidth}\raggedright\strut
\lstinline!-help!\strut
\end{minipage} & \begin{minipage}[t]{0.09\columnwidth}\raggedright\strut
\lstinline![bool]!\strut
\end{minipage} & \begin{minipage}[t]{0.56\columnwidth}\raggedright\strut
To activate helping messages. Gives description and list of available
flags.\strut
\end{minipage}\tabularnewline
\begin{minipage}[t]{0.26\columnwidth}\raggedright\strut
\lstinline!-debug!\strut
\end{minipage} & \begin{minipage}[t]{0.09\columnwidth}\raggedright\strut
\lstinline![bool]!\strut
\end{minipage} & \begin{minipage}[t]{0.56\columnwidth}\raggedright\strut
To activate live plot while PSD runs. Development flag.\strut
\end{minipage}\tabularnewline
\begin{minipage}[t]{0.26\columnwidth}\raggedright\strut
\lstinline!-useGFP!\strut
\end{minipage} & \begin{minipage}[t]{0.09\columnwidth}\raggedright\strut
\lstinline![bool]!\strut
\end{minipage} & \begin{minipage}[t]{0.56\columnwidth}\raggedright\strut
To activate use of GoFastPlugins. A suite of C++ plugins.\strut
\end{minipage}\tabularnewline
\begin{minipage}[t]{0.26\columnwidth}\raggedright\strut
\lstinline!-useRCM!\strut
\end{minipage} & \begin{minipage}[t]{0.09\columnwidth}\raggedright\strut
\lstinline![bool]!\strut
\end{minipage} & \begin{minipage}[t]{0.56\columnwidth}\raggedright\strut
Activate mesh level renumbering: Reverse Cuthill Mckee.\strut
\end{minipage}\tabularnewline
\begin{minipage}[t]{0.26\columnwidth}\raggedright\strut
\lstinline!-pipegnu!\strut
\end{minipage} & \begin{minipage}[t]{0.09\columnwidth}\raggedright\strut
\lstinline![bool]!\strut
\end{minipage} & \begin{minipage}[t]{0.56\columnwidth}\raggedright\strut
Use to activate real time pipe plotting using
\href{http://www.gnuplot.info/}{gnuplot}.\strut
\end{minipage}\tabularnewline
\begin{minipage}[t]{0.26\columnwidth}\raggedright\strut
\lstinline!-timelog!\strut
\end{minipage} & \begin{minipage}[t]{0.09\columnwidth}\raggedright\strut
\lstinline![bool]!\strut
\end{minipage} & \begin{minipage}[t]{0.56\columnwidth}\raggedright\strut
To activate time logging the different phases of the solver.\strut
\end{minipage}\tabularnewline
\begin{minipage}[t]{0.26\columnwidth}\raggedright\strut
\lstinline!-supercomp!\strut
\end{minipage} & \begin{minipage}[t]{0.09\columnwidth}\raggedright\strut
\lstinline![bool]!\strut
\end{minipage} & \begin{minipage}[t]{0.56\columnwidth}\raggedright\strut
Use when using a super computer without Xterm support\strut
\end{minipage}\tabularnewline
\begin{minipage}[t]{0.26\columnwidth}\raggedright\strut
\lstinline!-useMfront!\strut
\end{minipage} & \begin{minipage}[t]{0.09\columnwidth}\raggedright\strut
\lstinline![bool]!\strut
\end{minipage} & \begin{minipage}[t]{0.56\columnwidth}\raggedright\strut
Activate Mfornt interface for PSD.\strut
\end{minipage}\tabularnewline
\begin{minipage}[t]{0.26\columnwidth}\raggedright\strut
\lstinline!-bodyforce!\strut
\end{minipage} & \begin{minipage}[t]{0.09\columnwidth}\raggedright\strut
\lstinline![bool]!\strut
\end{minipage} & \begin{minipage}[t]{0.56\columnwidth}\raggedright\strut
To activate volumetric source term (body force).\strut
\end{minipage}\tabularnewline
\begin{minipage}[t]{0.26\columnwidth}\raggedright\strut
\lstinline!-vectorial!\strut
\end{minipage} & \begin{minipage}[t]{0.09\columnwidth}\raggedright\strut
\lstinline![bool]!\strut
\end{minipage} & \begin{minipage}[t]{0.56\columnwidth}\raggedright\strut
To use vectorial finite element method.\strut
\end{minipage}\tabularnewline
\begin{minipage}[t]{0.26\columnwidth}\raggedright\strut
\lstinline!-pointprobe!\strut
\end{minipage} & \begin{minipage}[t]{0.09\columnwidth}\raggedright\strut
\lstinline![bool]!\strut
\end{minipage} & \begin{minipage}[t]{0.56\columnwidth}\raggedright\strut
To postprocess point fields.\strut
\end{minipage}\tabularnewline
\begin{minipage}[t]{0.26\columnwidth}\raggedright\strut
\lstinline!-sequential!\strut
\end{minipage} & \begin{minipage}[t]{0.09\columnwidth}\raggedright\strut
\lstinline![bool]!\strut
\end{minipage} & \begin{minipage}[t]{0.56\columnwidth}\raggedright\strut
To solve via a sequential solver.\strut
\end{minipage}\tabularnewline
\begin{minipage}[t]{0.26\columnwidth}\raggedright\strut
\lstinline!-energydecomp!\strut
\end{minipage} & \begin{minipage}[t]{0.09\columnwidth}\raggedright\strut
\lstinline![bool]!\strut
\end{minipage} & \begin{minipage}[t]{0.56\columnwidth}\raggedright\strut
To activate energy decomposition, only for phase-field.\strut
\end{minipage}\tabularnewline
\begin{minipage}[t]{0.26\columnwidth}\raggedright\strut
\lstinline!-doublecouple!\strut
\end{minipage} & \begin{minipage}[t]{0.09\columnwidth}\raggedright\strut
\lstinline![bool]!\strut
\end{minipage} & \begin{minipage}[t]{0.56\columnwidth}\raggedright\strut
To activate double couple source for soildynamics.\strut
\end{minipage}\tabularnewline
\begin{minipage}[t]{0.26\columnwidth}\raggedright\strut
\lstinline!-constrainHPF!\strut
\end{minipage} & \begin{minipage}[t]{0.09\columnwidth}\raggedright\strut
\lstinline![bool]!\strut
\end{minipage} & \begin{minipage}[t]{0.56\columnwidth}\raggedright\strut
To use constrain condition in hybrid phase-field model.\strut
\end{minipage}\tabularnewline
\begin{minipage}[t]{0.26\columnwidth}\raggedright\strut
\lstinline!-top2vol-meshing!\strut
\end{minipage} & \begin{minipage}[t]{0.09\columnwidth}\raggedright\strut
\lstinline![bool]!\strut
\end{minipage} & \begin{minipage}[t]{0.56\columnwidth}\raggedright\strut
Activate top-ii-vol point source meshing for soil-dynamics.\strut
\end{minipage}\tabularnewline
\begin{minipage}[t]{0.26\columnwidth}\raggedright\strut
\lstinline!-getreactionforce!\strut
\end{minipage} & \begin{minipage}[t]{0.09\columnwidth}\raggedright\strut
\lstinline![bool]!\strut
\end{minipage} & \begin{minipage}[t]{0.56\columnwidth}\raggedright\strut
Activate routine for extraction reactions at surface.\strut
\end{minipage}\tabularnewline
\begin{minipage}[t]{0.26\columnwidth}\raggedright\strut
\lstinline!-plotreactionforce!\strut
\end{minipage} & \begin{minipage}[t]{0.09\columnwidth}\raggedright\strut
\lstinline![bool]!\strut
\end{minipage} & \begin{minipage}[t]{0.56\columnwidth}\raggedright\strut
Activate realtime pipe plotting using GnuPlot.\strut
\end{minipage}\tabularnewline
\begin{minipage}[t]{0.26\columnwidth}\raggedright\strut
\lstinline!-withmaterialtensor!\strut
\end{minipage} & \begin{minipage}[t]{0.09\columnwidth}\raggedright\strut
\lstinline![bool]!\strut
\end{minipage} & \begin{minipage}[t]{0.56\columnwidth}\raggedright\strut
Activate material tensor for building stiffness matrix.\strut
\end{minipage}\tabularnewline
\begin{minipage}[t]{0.26\columnwidth}\raggedright\strut
\lstinline!-crackdirichletcondition!\strut
\end{minipage} & \begin{minipage}[t]{0.09\columnwidth}\raggedright\strut
\lstinline![bool]!\strut
\end{minipage} & \begin{minipage}[t]{0.56\columnwidth}\raggedright\strut
To activate pre-cracked surface Dirichlet.\strut
\end{minipage}\tabularnewline
\begin{minipage}[t]{0.26\columnwidth}\raggedright\strut
\strut
\end{minipage} & \begin{minipage}[t]{0.09\columnwidth}\raggedright\strut
\strut
\end{minipage} & \begin{minipage}[t]{0.56\columnwidth}\raggedright\strut
\strut
\end{minipage}\tabularnewline
\begin{minipage}[t]{0.26\columnwidth}\raggedright\strut
\textbf{Integer flags}\strut
\end{minipage} & \begin{minipage}[t]{0.09\columnwidth}\raggedright\strut
\strut
\end{minipage} & \begin{minipage}[t]{0.56\columnwidth}\raggedright\strut
These flags accept a integer value followed by the flag itself. These
integer values are used in PSD simulations for various
definitions.\strut
\end{minipage}\tabularnewline
\begin{minipage}[t]{0.26\columnwidth}\raggedright\strut
\strut
\end{minipage} & \begin{minipage}[t]{0.09\columnwidth}\raggedright\strut
\strut
\end{minipage} & \begin{minipage}[t]{0.56\columnwidth}\raggedright\strut
\strut
\end{minipage}\tabularnewline
\begin{minipage}[t]{0.26\columnwidth}\raggedright\strut
\lstinline!-dirichletpointconditions!\strut
\end{minipage} & \begin{minipage}[t]{0.09\columnwidth}\raggedright\strut
\lstinline![int]!\strut
\end{minipage} & \begin{minipage}[t]{0.56\columnwidth}\raggedright\strut
Number of Dirichlet points.\strut
\end{minipage}\tabularnewline
\begin{minipage}[t]{0.26\columnwidth}\raggedright\strut
\lstinline!-dirichletconditions!\strut
\end{minipage} & \begin{minipage}[t]{0.09\columnwidth}\raggedright\strut
\lstinline![int]!\strut
\end{minipage} & \begin{minipage}[t]{0.56\columnwidth}\raggedright\strut
Number of Dirichlet boundaries.\strut
\end{minipage}\tabularnewline
\begin{minipage}[t]{0.26\columnwidth}\raggedright\strut
\lstinline!-bodyforceconditions!\strut
\end{minipage} & \begin{minipage}[t]{0.09\columnwidth}\raggedright\strut
\lstinline![int]!\strut
\end{minipage} & \begin{minipage}[t]{0.56\columnwidth}\raggedright\strut
Number of regions acted upon by bodyforce.\strut
\end{minipage}\tabularnewline
\begin{minipage}[t]{0.26\columnwidth}\raggedright\strut
\lstinline!-tractionconditions!\strut
\end{minipage} & \begin{minipage}[t]{0.09\columnwidth}\raggedright\strut
\lstinline![int]!\strut
\end{minipage} & \begin{minipage}[t]{0.56\columnwidth}\raggedright\strut
Number of Neumann/traction boundaries.\strut
\end{minipage}\tabularnewline
\begin{minipage}[t]{0.26\columnwidth}\raggedright\strut
\lstinline!-parmetis_worker!\strut
\end{minipage} & \begin{minipage}[t]{0.09\columnwidth}\raggedright\strut
\lstinline![int]!\strut
\end{minipage} & \begin{minipage}[t]{0.56\columnwidth}\raggedright\strut
Number of parallel workers used by ParMetis for partitioning.\strut
\end{minipage}\tabularnewline
\begin{minipage}[t]{0.26\columnwidth}\raggedright\strut
\lstinline!-lagrange!\strut
\end{minipage} & \begin{minipage}[t]{0.09\columnwidth}\raggedright\strut
\lstinline![int]!\strut
\end{minipage} & \begin{minipage}[t]{0.56\columnwidth}\raggedright\strut
Lagrange order used for FE spaces. 1 for P1 or 2 for P2.\strut
\end{minipage}\tabularnewline
\begin{minipage}[t]{0.26\columnwidth}\raggedright\strut
\lstinline!-dimension!\strut
\end{minipage} & \begin{minipage}[t]{0.09\columnwidth}\raggedright\strut
\lstinline![int]!\strut
\end{minipage} & \begin{minipage}[t]{0.56\columnwidth}\raggedright\strut
Accepts values 2 or 3. Use 3 for 3D. and 2 for 2D problem.\strut
\end{minipage}\tabularnewline
\begin{minipage}[t]{0.26\columnwidth}\raggedright\strut
\strut
\end{minipage} & \begin{minipage}[t]{0.09\columnwidth}\raggedright\strut
\strut
\end{minipage} & \begin{minipage}[t]{0.56\columnwidth}\raggedright\strut
\strut
\end{minipage}\tabularnewline
\begin{minipage}[t]{0.26\columnwidth}\raggedright\strut
\textbf{String flags}\strut
\end{minipage} & \begin{minipage}[t]{0.09\columnwidth}\raggedright\strut
\strut
\end{minipage} & \begin{minipage}[t]{0.56\columnwidth}\raggedright\strut
These flags accept a string value followed by the flag itself. These
string values are used in PSD simulations for various definitions.\strut
\end{minipage}\tabularnewline
\begin{minipage}[t]{0.26\columnwidth}\raggedright\strut
\strut
\end{minipage} & \begin{minipage}[t]{0.09\columnwidth}\raggedright\strut
\strut
\end{minipage} & \begin{minipage}[t]{0.56\columnwidth}\raggedright\strut
\strut
\end{minipage}\tabularnewline
\begin{minipage}[t]{0.26\columnwidth}\raggedright\strut
\lstinline!-mesh!\strut
\end{minipage} & \begin{minipage}[t]{0.09\columnwidth}\raggedright\strut
\lstinline![sting]!\strut
\end{minipage} & \begin{minipage}[t]{0.56\columnwidth}\raggedright\strut
Provide mesh to be solved by \lstinline!PSD_Solve!.\strut
\end{minipage}\tabularnewline
\begin{minipage}[t]{0.26\columnwidth}\raggedright\strut
\lstinline!-timediscretization!\strut
\end{minipage} & \begin{minipage}[t]{0.09\columnwidth}\raggedright\strut
\lstinline![sting]!\strut
\end{minipage} & \begin{minipage}[t]{0.56\columnwidth}\raggedright\strut
Time discretization type. Use ``generalized\_alpha'' or
``newmark\_beta'' or ``hht\_alpha'' or ``central\_difference''\strut
\end{minipage}\tabularnewline
\begin{minipage}[t]{0.26\columnwidth}\raggedright\strut
\lstinline!-nonlinearmethod!\strut
\end{minipage} & \begin{minipage}[t]{0.09\columnwidth}\raggedright\strut
\lstinline![sting]!\strut
\end{minipage} & \begin{minipage}[t]{0.56\columnwidth}\raggedright\strut
Nonlinear method type. Use ``Picard'' or ``Newton\_Raphsons''.\strut
\end{minipage}\tabularnewline
\begin{minipage}[t]{0.26\columnwidth}\raggedright\strut
\lstinline!-reactionforce!\strut
\end{minipage} & \begin{minipage}[t]{0.09\columnwidth}\raggedright\strut
\lstinline![sting]!\strut
\end{minipage} & \begin{minipage}[t]{0.56\columnwidth}\raggedright\strut
Reaction force calculation method ``stress\_based'' or
``variational\_based''.\strut
\end{minipage}\tabularnewline
\begin{minipage}[t]{0.26\columnwidth}\raggedright\strut
\lstinline!-doublecouple!\strut
\end{minipage} & \begin{minipage}[t]{0.09\columnwidth}\raggedright\strut
\lstinline![sting]!\strut
\end{minipage} & \begin{minipage}[t]{0.56\columnwidth}\raggedright\strut
Soil dynamics double couple. Use ``force\_based'' or
``displacement\_based''.\strut
\end{minipage}\tabularnewline
\begin{minipage}[t]{0.26\columnwidth}\raggedright\strut
\lstinline!-postprocess!\strut
\end{minipage} & \begin{minipage}[t]{0.09\columnwidth}\raggedright\strut
\lstinline![sting]!\strut
\end{minipage} & \begin{minipage}[t]{0.56\columnwidth}\raggedright\strut
To communicate what to postprocess ``u'', ``v'', ``a'', ``uv'' , ``ud'',
``ua'', ``d'', ``ud'', or ``uav''.\strut
\end{minipage}\tabularnewline
\begin{minipage}[t]{0.26\columnwidth}\raggedright\strut
\lstinline!-partitioner!\strut
\end{minipage} & \begin{minipage}[t]{0.09\columnwidth}\raggedright\strut
\lstinline![sting]!\strut
\end{minipage} & \begin{minipage}[t]{0.56\columnwidth}\raggedright\strut
Mesh partitioner could be
``\href{http://glaros.dtc.umn.edu/gkhome/metis/metis/overview}{metis}''
``\href{http://glaros.dtc.umn.edu/gkhome/metis/parmetis/overview}{parmetis}''
or ``\href{http://www.labri.fr/perso/pelegrin/scotch/}{scotch}''.\strut
\end{minipage}\tabularnewline
\begin{minipage}[t]{0.26\columnwidth}\raggedright\strut
\lstinline!-problem!\strut
\end{minipage} & \begin{minipage}[t]{0.09\columnwidth}\raggedright\strut
\lstinline![sting]!\strut
\end{minipage} & \begin{minipage}[t]{0.56\columnwidth}\raggedright\strut
Interested problem. Use ``linear\_elasticity'', ``damage'',
``elastodynamics'', or ``soildynamics''.\strut
\end{minipage}\tabularnewline
\begin{minipage}[t]{0.26\columnwidth}\raggedright\strut
\lstinline!-model!\strut
\end{minipage} & \begin{minipage}[t]{0.09\columnwidth}\raggedright\strut
\lstinline![sting]!\strut
\end{minipage} & \begin{minipage}[t]{0.56\columnwidth}\raggedright\strut
Interested model. Use ``hybrid\_phase\_field'' or ``Mazar''.\strut
\end{minipage}\tabularnewline
\bottomrule
\end{longtable}

\subsection{Configuration flags}

These are a set of commandline flags/options that control your PSD
configuration via the automake ligo.

\begin{longtable}[]{@{}lll@{}}
\toprule
\begin{minipage}[b]{0.14\columnwidth}\raggedright\strut
\textbf{Flag}\strut
\end{minipage} & \begin{minipage}[b]{0.39\columnwidth}\raggedright\strut
\textbf{Description}\strut
\end{minipage} & \begin{minipage}[b]{0.39\columnwidth}\raggedright\strut
\textbf{Examples}\strut
\end{minipage}\tabularnewline
\midrule
\endhead
\begin{minipage}[t]{0.14\columnwidth}\raggedright\strut
\lstinline!--prefix!\strut
\end{minipage} & \begin{minipage}[t]{0.39\columnwidth}\raggedright\strut
Enter the directory where you wish to install PSD.Note that you will
need to have read and write permission for this directory.\emph{This
flag is an optional flag}\strut
\end{minipage} & \begin{minipage}[t]{0.39\columnwidth}\raggedright\strut
\lstinline!--prefix=/usr! \lstinline!--prefix=/usr/local!
\lstinline!--prefix=/home/install!\strut
\end{minipage}\tabularnewline
\begin{minipage}[t]{0.14\columnwidth}\raggedright\strut
\lstinline!--with-FreeFEM!\strut
\end{minipage} & \begin{minipage}[t]{0.39\columnwidth}\raggedright\strut
Enter the directory where FreeFem binary has been installed. Tip, in
your terminal \lstinline!which FreeFem++! can help you find this
directory.\emph{This flag is an optional flag}\strut
\end{minipage} & \begin{minipage}[t]{0.39\columnwidth}\raggedright\strut
\lstinline!--with-FreeFEM=/usr/bin!
\lstinline!--with-FreeFEM=/home/install/bin!
\lstinline!--with-FreeFEM=/usr/local/bin!\strut
\end{minipage}\tabularnewline
\begin{minipage}[t]{0.14\columnwidth}\raggedright\strut
\lstinline!--with-Gmsh!\strut
\end{minipage} & \begin{minipage}[t]{0.39\columnwidth}\raggedright\strut
Enter the directory where Gmsh binary has been installed. Tip, in your
terminal \lstinline!which gmsh! can help you find
thisdirectory.\emph{This flag is an optional flag}\strut
\end{minipage} & \begin{minipage}[t]{0.39\columnwidth}\raggedright\strut
\lstinline!--with-Gmsh=/usr/bin!
\lstinline!--with-Gmsh=/home/install/bin!
\lstinline!--with-Gmsh=/usr/local/bin!\strut
\end{minipage}\tabularnewline
\begin{minipage}[t]{0.14\columnwidth}\raggedright\strut
\lstinline!--with-mgis!\strut
\end{minipage} & \begin{minipage}[t]{0.39\columnwidth}\raggedright\strut
Enter the directory where Mgis has been installed. \emph{This flag is an
optional flag}\strut
\end{minipage} & \begin{minipage}[t]{0.39\columnwidth}\raggedright\strut
\lstinline!--with-mgis=/usr! \lstinline!--with-mgis=/home/install!
\lstinline!--with-mgis=/usr/local!\strut
\end{minipage}\tabularnewline
\begin{minipage}[t]{0.14\columnwidth}\raggedright\strut
\lstinline!--with-salome!\strut
\end{minipage} & \begin{minipage}[t]{0.39\columnwidth}\raggedright\strut
Enter the directory where SALOME has been installed. \emph{This flag is
an optional flag}\strut
\end{minipage} & \begin{minipage}[t]{0.39\columnwidth}\raggedright\strut
\lstinline!--with-salome=/home/SALOME-UB22.04!
\lstinline!--with-salome=/home/install/SALOME-UB22.04!
\lstinline!--with-salome=/usr/local/SALOME-UB22.04!\strut
\end{minipage}\tabularnewline
\begin{minipage}[t]{0.14\columnwidth}\raggedright\strut
\lstinline!--with-mfront!\strut
\end{minipage} & \begin{minipage}[t]{0.39\columnwidth}\raggedright\strut
Enter the directory where Mfront binary has been installed. \emph{This
flag is an optional flag}\strut
\end{minipage} & \begin{minipage}[t]{0.39\columnwidth}\raggedright\strut
\lstinline!--with-mfront=/usr/bin!
\lstinline!--with-mfront=/home/install/bin!
\lstinline!--with-mfront=/usr/local/bin!\strut
\end{minipage}\tabularnewline
\begin{minipage}[t]{0.14\columnwidth}\raggedright\strut
\lstinline!--with-dependencies!\strut
\end{minipage} & \begin{minipage}[t]{0.39\columnwidth}\raggedright\strut
Enter yes or no as an option to this flag, default is no. If yesis
entered to this command, PSD will build and compile itsdependencies for
you. If yes PSD will compile PETSc, FreeFEM,Mgis, MFront, Metis,
ParMetis, Scalapack, mumps, hpddm,slepc, suitsspars, tetgen.\emph{This
flag is an optional flag}\strut
\end{minipage} & \begin{minipage}[t]{0.39\columnwidth}\raggedright\strut
\lstinline!--with-dependencies=yes! \lstinline!--with-dependencies=no!
\strut
\end{minipage}\tabularnewline
\begin{minipage}[t]{0.14\columnwidth}\raggedright\strut
\lstinline!--with-zipped_dependencies!\strut
\end{minipage} & \begin{minipage}[t]{0.39\columnwidth}\raggedright\strut
Enter yes or no as an option to this flag, default is no. If yesis
entered to this command, PSD will look for \lstinline!.tar.gz! files for
dependencies in \lstinline!ext! folder and compile themfor you. If yes
PSD will expect \lstinline!.tar.gz! for PETSc, FreeFEM,Mgis, MFront,
Metis, ParMetis, Scalapack, mumps, hpddm,slepc, suitsspars, tetgen from
ext folder.\emph{This flag is an optional flag}\strut
\end{minipage} & \begin{minipage}[t]{0.39\columnwidth}\raggedright\strut
\lstinline!--with-zipped_dependencies=yes!
\lstinline!--with-zipped_dependencies=no! \strut
\end{minipage}\tabularnewline
\bottomrule
\end{longtable}

\subsection{make options for PSD}

Once \lstinline!./configure! runs successfully your Makefiles will be
generated thanks to automake. Different options are available with
\lstinline!make! command some are native to Make (still listed here,
sorry to my linux co-geeks)

​

\begin{longtable}[]{@{}lll@{}}
\toprule
\begin{minipage}[b]{0.15\columnwidth}\raggedright\strut
\textbf{Command}\strut
\end{minipage} & \begin{minipage}[b]{0.47\columnwidth}\raggedright\strut
\textbf{Description}\strut
\end{minipage} & \begin{minipage}[b]{0.30\columnwidth}\raggedright\strut
\textbf{Example}\strut
\end{minipage}\tabularnewline
\midrule
\endhead
\begin{minipage}[t]{0.15\columnwidth}\raggedright\strut
\lstinline!make!\strut
\end{minipage} & \begin{minipage}[t]{0.47\columnwidth}\raggedright\strut
Command responsible to compile PSD for you. This is necessary.\strut
\end{minipage} & \begin{minipage}[t]{0.30\columnwidth}\raggedright\strut
\lstinline!make!\strut
\end{minipage}\tabularnewline
\begin{minipage}[t]{0.15\columnwidth}\raggedright\strut
\lstinline!-j4!\strut
\end{minipage} & \begin{minipage}[t]{0.47\columnwidth}\raggedright\strut
Activates parallel make, i.e., faster compilation on 4 cores.\emph{This
flag is an optional flag}\strut
\end{minipage} & \begin{minipage}[t]{0.30\columnwidth}\raggedright\strut
\lstinline!make -j4!\strut
\end{minipage}\tabularnewline
\begin{minipage}[t]{0.15\columnwidth}\raggedright\strut
\lstinline!install!\strut
\end{minipage} & \begin{minipage}[t]{0.47\columnwidth}\raggedright\strut
Command that installs PSD for you, this command should follow the
\lstinline!make! command.\strut
\end{minipage} & \begin{minipage}[t]{0.30\columnwidth}\raggedright\strut
\lstinline!make install! \lstinline!make install -j4!\strut
\end{minipage}\tabularnewline
\begin{minipage}[t]{0.15\columnwidth}\raggedright\strut
\lstinline!check!\strut
\end{minipage} & \begin{minipage}[t]{0.47\columnwidth}\raggedright\strut
Command that should follow \lstinline!make install! helps to check the
PSD installation.\emph{This command is an optional but
recommended}\strut
\end{minipage} & \begin{minipage}[t]{0.30\columnwidth}\raggedright\strut
\lstinline!make check!\strut
\end{minipage}\tabularnewline
\begin{minipage}[t]{0.15\columnwidth}\raggedright\strut
\lstinline!clean!\strut
\end{minipage} & \begin{minipage}[t]{0.47\columnwidth}\raggedright\strut
Command that cleans PSD's compilation directory.\emph{This command is an
optional}\strut
\end{minipage} & \begin{minipage}[t]{0.30\columnwidth}\raggedright\strut
\lstinline!make clean!\strut
\end{minipage}\tabularnewline
\begin{minipage}[t]{0.15\columnwidth}\raggedright\strut
\lstinline!maintainer-clean!\strut
\end{minipage} & \begin{minipage}[t]{0.47\columnwidth}\raggedright\strut
Command that cleans PSD's compilation directory throughly.\emph{This
command is an optional}\strut
\end{minipage} & \begin{minipage}[t]{0.30\columnwidth}\raggedright\strut
\lstinline!make maintainer-clean!\strut
\end{minipage}\tabularnewline
\begin{minipage}[t]{0.15\columnwidth}\raggedright\strut
\lstinline!tutorials!\strut
\end{minipage} & \begin{minipage}[t]{0.47\columnwidth}\raggedright\strut
Command that builds PSD tutorials in \lstinline!$HOME! directory. This
should follow/be-used only after \lstinline!make install!.\emph{This
command is an optional}\strut
\end{minipage} & \begin{minipage}[t]{0.30\columnwidth}\raggedright\strut
\lstinline!make tutorials!\strut
\end{minipage}\tabularnewline
\begin{minipage}[t]{0.15\columnwidth}\raggedright\strut
\lstinline!install-devl!\strut
\end{minipage} & \begin{minipage}[t]{0.47\columnwidth}\raggedright\strut
Command that installs developers version of PSD for you, this command
should follow the \lstinline!make! command.\emph{This command is an
optional}\strut
\end{minipage} & \begin{minipage}[t]{0.30\columnwidth}\raggedright\strut
\lstinline!make install-devel!\strut
\end{minipage}\tabularnewline
\begin{minipage}[t]{0.15\columnwidth}\raggedright\strut
\lstinline!documentation!\strut
\end{minipage} & \begin{minipage}[t]{0.47\columnwidth}\raggedright\strut
Command that builds documentation, in html, and pdf formats. This
command should follow the \lstinline!make! command. Note that this needs
pandoc installed in your system.\emph{This command is an optional}\strut
\end{minipage} & \begin{minipage}[t]{0.30\columnwidth}\raggedright\strut
\lstinline!make documentation!\strut
\end{minipage}\tabularnewline
\bottomrule
\end{longtable}

\emph{To report bugs, issues, feature-requests contact:}

\begin{itemize}
\tightlist
\item
  \textbf{mohd-afeef.badri@cea.fr}
\item
  \textbf{mohd-afeef.badri@hotmail.com}
\end{itemize}
